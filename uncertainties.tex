\chapter{Systematic Uncertainties}\label{sec:uncertainties}

The uncertainties associated with the choice of background shape are incorporated into the fit to the data through the use of the discrete profiling method.
They are, therefore, reflected in the statistical uncertainties obtained from the fit.
The systematic uncertainties, affecting either the normalization or the shape of the signal expectation, are listed below, and the numerical values are summarized in Table~\ref{tab:syst}, which also indicates whether the effect is correlated between the data-taking periods. Overall, systematic uncertainty comprises only about 5\% of the total uncertainty in the measurement, as the statistical uncertainty due to the small number of expected signal events dominates. Observed and expected nuisance parameter impacts on the signal strength are included in Appendix \ref{sec:appendix_impacts}. Nuisance parameter impacts corresponding to the $\brzg/\brgg$ measurement are also included. 
\begin{itemize}
  \item Theoretical cross section calculations: These include the effects of the choice of PDFs, the value of the strong coupling constant (\alpS), and the effect of missing higher orders in the perturbative cross section calculations, evaluated from variations of the renormalization and factorization scales ($\mu_{\mathrm{R}}$, $\mu_{\mathrm{F}}$)~\cite{cite:cs1,cite:cs2,Butterworth:2015oua}. The uncertainties are treated as independent for each Higgs boson production mechanism. The uncertainty in $\brzg$ is also considered~\cite{LHC-YR4}.
  \item Underlying event and parton shower modeling: The uncertainty associated with the choice and tuning of the generator is estimated with dedicated samples which are generated by
  varying the parameters of the tune used to generate the original signal samples. The uncertainties are treated as correlated for the 2017 and 2018 samples, which use the CP5 tune~\cite{Sirunyan:2019dfx}, while being uncorrelated with the 2016 sample, which uses the CUETP8M1 tune~\cite{Khachatryan:2015pea}.

  \item Integrated luminosity:
  The integrated luminosities for the 2016, 2017, and 2018 data-taking years have uncertainties of 1.2\%, 2.3\%, and 2.5\%~\cite{CMS-LUM-17-003,LUM-17-004,LUM-18-002}, respectively, corresponding to an overall uncertainty for the 2016--2018 period of 1.6\%, the improvement in precision reflecting the (uncorrelated) time evolution of some systematic effects.

  \item L1 trigger: During the 2016 and 2017 data-taking periods, a gradual shift in the timing of the inputs of the ECAL L1 trigger in the $\abs{\eta} > 2.4$ region led to a specific inefficiency. A correction of approximately 1\% is applied to the simulation along with the corresponding uncertainty in the inefficiency measurement.

  \item Trigger: Uncertainties are evaluated for the corrections applied to the simulation to match the trigger efficiencies measured in data with $\PZ\to\epem$ and $\PZ\to\mpmm$ events.

  \item Photon identification and isolation: Uncertainties are evaluated for the corrections applied to the simulation to match the selection efficiencies in data measured with $\PZ\to\epem$ events.

  \item Lepton identification and isolation: Uncertainties are evaluated for the corrections applied to the simulation to match electron and muon selection efficiencies in data measured with $\PZ\to\epem$ and $\PZ\to\mpmm$ events.

  \item Pileup modeling: The uncertainty in the description of the pileup in the signal simulation is estimated by varying the total inelastic cross section by $\pm4.6$\%~\cite{Sirunyan:2018nqx}.

 \item Kinematic BDT: The uncertainties in the photon and lepton energy and the correction of the photon MVA discriminant are propagated to $\Dkin$. Changes in $\Dkin$ cause the migration of signal events across category boundaries.

  \item VBF BDT: The uncertainties in the jet energy and the uncertainty in $\Dkin$ are propagated to $\DVBF$. Changes in $\DVBF$ cause the migration of signal events across category boundaries.

  \item Photon energy scale and resolution:
	The photon energy in the simulation is varied due to the ECAL energy scale and resolution uncertainties, and the effects on the signal mean and resolution parameters are propagated to the fits.

  \item Lepton momentum scale and resolution:
  The lepton momentum in the simulation is varied due to the lepton momentum scale and resolution uncertainties, and the effects on signal mean and resolution parameters are propagated to the fits.
\end{itemize}
In the $\brzg/\brgg$ measurement, the common sources of theoretical and systematic uncertainty in the two analyses are treated as correlated in the fit. These are the theoretical uncertainties in the Higgs production cross section calculations, and the systematic uncertainties in the underlying event and parton shower modeling, the integrated luminosity, and the L1 trigger inefficiency.
The remaining uncertainties are treated as uncorrelated.
\begin{table*}[tb!]
  \centering
  \caption{Sources of systematic uncertainty affecting the simulated signal. The normalization effect on the expected yield, or the effect on the signal shape parameters, is given as indicated, with the values averaged over all event categories. The third column shows the uncertainties that have a correlated effect across the three data-taking periods.
    \label{tab:syst}}
    {
  \begin{tabular}{l@{\hskip 0.3in}c@{\hskip 0.3in}c}
  \hline
  Sources                &  Uncertainty (\%)   & Year-to-year correlation  \\ \hline
  \multicolumn{3}{c}{\textbf{Normalization}}\\
  Theoretical &    &  \\
  -- $\brzg$ & 5.7    &  Yes\\
  -- $\Pg\Pg\PH$ cross section ($\mu_{\mathrm{F}}$, $\mu_{\mathrm{R}}$) & 3.9 & Yes\\
  -- $\Pg\Pg\PH$ cross section (\alpS)& 2.6& Yes\\
  -- $\Pg\Pg\PH$ cross section (PDF)& 1.9 &Yes\\
  -- VBF cross section ($\mu_{\mathrm{F}}$, $\mu_{\mathrm{R}}$)&  0.4 & Yes\\
  -- VBF cross section (\alpS)&  0.5 & Yes\\
  -- VBF cross section (PDF)&  2.1 &Yes\\
  -- $\PW\PH$ cross section ($\mu_{\mathrm{F}}$, $\mu_{\mathrm{R}}$) & $^{+0.6}_{-0.7}$ &  Yes\\
  -- $\PW\PH$ cross section (PDF)& 1.7 & Yes\\
  -- $\PZ\PH$ cross section ($\mu_{\mathrm{F}}$, $\mu_{\mathrm{R}}$) & $^{+3.8}_{-3.1}$ &Yes \\
  -- $\PZ\PH$ cross section (PDF)& 1.3  &   Yes\\
  -- $\PW\PH$/$\PZ\PH$  cross section (\alpS) & 0.9 &Yes\\
  -- $\ttbar\PH$ cross section ($\mu_{\mathrm{F}}$, $\mu_{\mathrm{R}}$)& $^{+5.8}_{-9.2}$ &  Yes\\
  -- $\ttbar\PH$ cross section (\alpS)& 2.0 & Yes \\
  -- $\ttbar\PH$ cross section (PDF)& 3.0 &Yes \\
  Underlying event and parton shower   	      & 3.7--4.4             &       	     Partial         \\
  Integrated luminosity  &  1.2--2.5 & Partial \\
  L1 trigger    	     &      	 	0.1--0.4      	 	   		 & No       \\
  Trigger                             						      &             	  &				  \\
  
  -- Electron  channel                  				                      				    &0.9--1.9	  &     No      \\
  -- Muon channel                       					                     					&0.1--0.4  	 &    No    \\
  Photon identification and isolation                      &    0.2--5.0            			 &  Yes           \\
  Lepton identification and isolation      &       	 	   & \\
  
  -- Electron channel                   			   	                     					&0.5--0.7  	 	 &      Yes      \\
  -- Muon channel                       				   	       					  			&0.3--0.4  	 &          Yes     \\
  Pileup                 				   			   & 0.4--1.0 & Yes            \\
  Kinematic BDT               & 2.5--3.7         				  &  Yes\\
  VBF BDT                            				 			   & 5.9--14.0 &  Yes\\
  \multicolumn{3}{c}{\textbf{Shape parameters}}\\
  Photon energy and momentum					       &&             \\
  -- Signal mean                        						               					 & 0.1--0.4       &      Yes        \\
  -- Signal resolution                        						               		& 3.1--5.9       &      Yes        \\
  Lepton energy and momentum     				 			 				  	  &&          	                 \\
  -- Signal mean                        						       							 & 0.007  &     Yes   \\
  -- Signal resolution                  				  	     							   & 0.007--0.010 &     Yes          \\\hline
  \end{tabular}
  }
  \end{table*}
