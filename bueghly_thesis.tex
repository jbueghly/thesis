%%%%%%%%%%%%%%%%%%%%%%%%%%%%%%%%%%%%%%%%%%%%%%%%%%%%%%%%%%%%%%%%%%%%%%
% James Bueghly's Northwestern Ph.D. Thesis
% Department of Physics and Astronomy
% CMS Run II Higgs to Z+gamma Analysis
%%%%%%%%%%%%%%%%%%%%%%%%%%%%%%%%%%%%%%%%%%%%%%%%%%%%%%%%%%%%%%%%%%%%%%

%%%%%%%%%%%%%%%%%%%%%%%%%%%%%%%%%%%%%%%%%%%%%%%%%%%%%%%%%%%%%%%%%%%%%%
% template from Northwestern Department of Mathematics
% nuthesis-template.tex - Miguel A, Lerma - 4/23/2018
%                         mlerma@math.northwestern.edu
%%%%%%%%%%%%%%%%%%%%%%%%%%%%%%%%%%%%%%%%%%%%%%%%%%%%%%%%%%%%%%%%%%%%%%

% The nuthesis class is based on % amsbook.cls.
\documentclass[12pt]{template/nuthesis}	
\usepackage{xcolor}
\usepackage{graphicx}
\usepackage{subcaption}
\usepackage{multirow}
%\usepackage{notoccite}
\usepackage{cmsstyle/heppennames2}
\usepackage{cmsstyle/hepparticles}
\usepackage{cmsstyle/ptdr-definitions}
\newcommand\numberthis{\addtocounter{equation}{1}\tag{\theequation}}
\newcommand{\tabincell}[2]{\begin{tabular}{@{}#1@{}}#2\end{tabular}}
\newcommand{\LumiT}{138}
\newcommand{\Lumia}{36.3}
\newcommand{\Lumib}{41.5} 
\newcommand{\Lumic}{59.8}
\newcommand{\mH}{125.38}
\newcommand{\signalstrength}{\ensuremath{2.4\pm0.9}}
\newcommand{\signalstrengthExpanded}{\ensuremath{2.4^{+0.8}_{-0.9}}\,\stat\,^{+0.3}_{-0.2}\,\syst}
\newcommand{\signalstrengthdijetthree}{\ensuremath{12.3^{+3.7}_{-3.5}}}
\newcommand{\br}{$0.21\pm0.08$}
\newcommand{\brExpanded}{\ensuremath{0.21^{+0.07}_{-0.08}\,\stat\,^{+0.03}_{-0.02}\,\syst}}
\newcommand{\explimit}{1.8}
\newcommand{\obslimit}{4.1}
\newcommand{\expsig}{1.2}
\newcommand{\obssig}{2.7}
\newcommand{\compatibility}{1.6}
\newcommand{\brRatio}{\ensuremath{1.5^{+0.7}_{-0.6}}}
\newcommand{\brRatioCompat}{1.5}
\newcommand{\channelcompatp}{0.02}
\newcommand{\channelcompatsigma}{2.3}
\newcommand{\Dkin}{\ensuremath{\mathcal{D_{\mathrm{kin}}}}}
\newcommand{\DVBF}{\ensuremath{\mathcal{D}_{\mathrm{VBF}}}}

% shortcuts and values
\newcommand{\hzg}{$\PH\to\PZ\gamma$}
\newcommand{\hmumu}{$H \rightarrow \mu^{+}\mu^{-}$}
\newcommand{\hgg}{$\PH\to\gamma\gamma$}
\newcommand{\hzz}{$H \rightarrow ZZ$}
\newcommand{\lplm}{\ensuremath{\ell^{+}\ell^{-}}}
\newcommand{\epem}{\ensuremath{\Pe^{+}\Pe^{-}}}
\newcommand{\mpmm}{\ensuremath{\mu^{+}\mu^{-}}}
\newcommand{\ptone}{\ensuremath{p_{\mathrm{T1}}}\xspace}
\newcommand{\pttwo}{\ensuremath{p_{\mathrm{T2}}}\xspace}
\newcommand{\ptoner}{\ensuremath{p^{\mathrm{R}}_{\mathrm{T1}}}\xspace}
\newcommand{\pttwor}{\ensuremath{p^{\mathrm{R}}_{\mathrm{T2}}}\xspace}

\author{James Bueghly}

\title{Search for Higgs Boson Decays to a Z Boson and a Photon}

%\degree{DOCTOR OF PHILOSOPHY}  % Default: DOCTOR OF PHILOSOPHY

\field{Physics}            % Default: Mathematics

\graduationmonth{June}         % The default is June or December
                                % depending on current date.

\graduationyear{2022}          % Default: current year.


				% Use \includeonly to select the 
%\includeonly{chap1,chap2,...}	% chapters to include if you are 
				% using the \include command below.
				% This way you can latex only a the 
				% part you are working on, which 
				% is faster than latexing the entire 
				% thesis. 

\begin{document}
%	
%	THE BODY OF YOUR THESIS STARTS HERE
%

%%%%%%%%%%%%%%%%%%%%%%
% Some initial stuff %
%%%%%%%%%%%%%%%%%%%%%%

\frontmatter		% Preliminary pages start here.

\maketitle		% Produces the title page.

\copyrightpage		% Creates the copyright page.


% Abstract.
\abstract

Since its discovery in 2012 at the Large Hadron Collider (LHC), efforts have been made to measure and characterize the properties of the Higgs boson. 
Among these efforts have been searches for rare decays of the Higgs predicted by the standard model (SM) of particle physics. 
One such decay is the process \hzg, which has an expected branching fraction of $\mathcal{B}(\PH\to\PZ\gamma) = (1.57 \pm 0.09) \times 10^{-3}$ in the SM, 
assuming a Higgs boson mass of $m_\PH = \mH\GeV$.
This decay mode has not yet been experimentally observed, and its observation and measurement remains an important goal of Higgs physics research at the LHC.
In addition, a measurement of this decay mode at a rate deviating from the SM prediction would provide indirect evidence of new physics beyond the SM. 

This thesis presents a search for \hzg, where $\mathrm{Z}\to\ell^+\ell^-$ with $\ell=\mathrm{e}$ or $\mu$. The search is performed using a sample of proton-proton ($\mathrm{pp}$) collision data at a center-of-mass energy of $13~\mathrm{TeV}$, recorded by the Compact Muon Solenoid experiment at the LHC, corresponding to an integrated luminosity of $138~\mathrm{fb}^{-1}$. 
Events are assigned to mutually exclusive categories, which exploit differences in both event topology and kinematics of distinct Higgs production mechanisms to enhance signal sensitivity. 
To detect a potential signal, fits are performed to the distributions of ${\ell^+\ell^-\gamma}$ invariant mass in each of these categories simultaneously.
The signal strength $\mu$, defined as the product of the cross section and the branching fraction [$\sigma(\mathrm{pp}\to\mathrm{H})\mathcal{B}(\mathrm{H}\to\mathrm{Z}\gamma)$] relative to the SM expectation, is found to be $\mu= 2.4\,\pm0.9$ for $m_\PH = \mH\GeV$. This measurement corresponds to $\sigma(\mathrm{pp}\to\mathrm{H})\mathcal{B}(\mathrm{H}\to\mathrm{Z}\gamma)=0.21\pm0.08~\mathrm{pb}$. The statistical significance of the observed excess of events is $2.7$ standard deviations. The observed (expected) upper limit at $95$\% confidence level on $\mu$ is $4.1$\,($1.8$). 
The ratio of branching fractions $\mathcal{B}(\mathrm{H}\to\mathrm{Z}\gamma)/\mathcal{B}(\mathrm{H}\to\gamma\gamma)$ is measured to be $1.5^{+0.7}_{-0.6}$, which agrees with the SM prediction at the 
$1.5$ standard deviation level. 


\acknowledgements	% Acknowledgements (optional).

I would like to thank my adviser, Mayda Velasco, for the opportunity to pursue a goal I dreamed up years ago and for her support along the way. I'd also like to thank my many colleagues at Northwestern University, National Central University in Taiwan, CERN, and Fermilab who helped me grow both as a scientist and a person over the course of my Ph.D. 

Special thanks are due to those who significantly impacted the work contained in this thesis: My coanalyzer Ming-Yan Lee, whose discipline and tireless work ethic kept us all going through the many challenges we faced; postdoc Andrew Gilbert, whose expertise and guidance were critical; and the CMS leaders who guided and reviewed our work, including the Higgs conveners, Analysis Review Committee, and our colleague and language editor Jeff Richman. Without all of you, this thesis would not exist.

To my Northwestern colleagues: I will always be grateful for your friendship over these years. The moments we shared away from the work kept me sane, focused, and balanced. It is impossible to name all of them here, but the squash games with Nate Odell, the lunches with Nate, Brian Pollack, Thoth Gunter, and Joseph Cordero Mercado, and our adventures at COFI in San Juan were all very meaningful to me. 

Last, but not least, I want to thank my parents, sisters, and my partner Anna. You all have shown me so much love, encouragement, and patience throughout this process. For that, I feel extremely fortunate. This is for all of you. 



%\preface		% Preface (optional).
%
%This is the preface.


%% A few more optional pages (uncomment if needed)
%
%\listofabbreviations 
%
%This is the list of abbreviations (optional).
%
%\glossary
%
%This is the glossary (optional).
%
%\nomenclature
%
%This is the nomenclature (optional).
%
%% Note that the dedication text must be passed as an argument
%% of the \dedication command
%\dedication{This is the dedication (optional).}
%

\clearpage\phantomsection % needed for the hyperlinks to work correctly
\tableofcontents	% Table of Contents will be automatically
			% generated and placed here.

\clearpage\phantomsection % needed for the hyperlinks to work correctly
\listoftables		% List of Tables and List of Figures will be placed

\clearpage\phantomsection % needed for the hyperlinks to work correctly
\listoffigures		% here, if applicable (optional).



\mainmatter             % Actual text starts here.

%%%%%%%%%%%%%%%%%%%%%%%%%%%
% Actual text starts here %
%%%%%%%%%%%%%%%%%%%%%%%%%%%

% If there is an introduction it must be the first chapter

%\chapter{Introduction or Title of First Chapter}	% The first chapter.
%				% \chapter command is of the form
%				% \chapter[..]{..} or \chapter{..} where
%	... text ...		% {chapter heading} and [entry in table of
%				% contents].
%\section{First section of chapter 1}
%				% IMPORTANT: If your chapter heading consists
%				% of more than one lines, it will be auto-
%	... text ...		% matically broken into separate lines.
%				% However, if you don't like the way LaTeX
%				% breaks the chapter heading into lines, use
%\section{Another section}	% `\newheadline' command to break lines.
%				% Never use \\ in sectional (e.g., chapter,
%	... text ...		% section, subsection) headings.
%
%\chapter{Title of Second Chapter}	% Chapter 2.
%
%	... text ...
%
%\section{First section of chapter 2}
%
%	... text...
%
%\subsection{A subsection}
%
%	... more text ...
%
%\subsubsection{A subsubsection}
%
%	... more text ...


% Alternatively, you may write the chapters in separate
% files, say chap1.tex, chap2.tex, etc., and include them 
% with commands:

%\include{chap1}

%\include{chap2}

% The command \includeonly above allows to include a selected 
% set of chapters only.


\chapter{Introduction}

Since the discovery of the Higgs boson~\cite{Aad_2012,Chatrchyan_2012,CMS:2013btf} at the LHC, an extensive program of measurements~\cite{PhysRevD.98.030001} has been undertaken to determine its properties and couplings to different types of particles and to assess whether these properties are consistent with those predicted by the standard model (SM). With the successful running of the LHC, large data samples of proton-proton ($\Pp\Pp$) collisions at $\sqrt s = 13\TeV$ have been accumulated, increasing the sensitivity to rare decays of the Higgs boson. 
Such decays also provide probes for possible contributions arising from physics beyond the SM (BSM) and include the process 
$\PH \to \PZ \gamma$~\cite{Abba96, Chen12, Htollg-FB-Sun, Passarino, Campbell_2013hz, Degrassi:2019yix, Low:2011gn}.

Figure \ref{fig:fey} shows Feynman diagrams for the key SM contributions to the $\PH \to \PZ \gamma$ decay process. 
Experimentally, the final state resulting from $\PZ \to \ell^+ \ell^-$ ($\ell = \Pe$ or $\mu$) is the most accessible, since the leptons are highly distinctive, well-measured, and provide a means to trigger the recording of the events. 
\begin{figure*}[!b]
\includegraphics[width=0.9\textwidth]{fig/intro/Figure_001.pdf}
\caption{Feynman diagrams for $\PH\to\PZ\gamma$ decay.} \label{fig:fey}
\end{figure*}
In the SM, the expected branching fraction for $\PH\to\PZ\gamma$ is $\mathcal{B}(\PH\to\PZ\gamma) = (1.57 \pm 0.09) \times 10^{-3}$, assuming a Higgs boson mass of $m_\PH = \mH\GeV$. This branching fraction is comparable to $\mathcal{B}(\PH\to\gamma\gamma)  = (2.27 \pm 0.04) \times 10^{-3}$~\cite{LHC-YR4,CMS:2021kom}. The value $m_\PH=\mH \pm 0.14\GeV$ is taken from the most recent CMS Higgs boson mass measurement~\cite{CMS:2020xrn}, which uses the combination of $\PH\to\gamma\gamma$ and $\PH\to\PZ\PZ^*\to 4\ell$ results from the $2011$--$2012$ and $2016$ data samples.
The ratio $\mathcal{B}(\PH\rightarrow\PZ\gamma)/\mathcal{B}(\PH\rightarrow\gamma\gamma) = 0.69 \pm 0.04$ is 
potentially sensitive to BSM physics, such as supersymmetry and extended Higgs 
sectors~\cite{Djouadi:1996yq,Zg_theory_extension,Zg_theory_decaywidth,Chen:2013vi}.
The effects from these models can shift the $\PH\rightarrow\PZ\gamma$ and $\PH\to\PGg\PGg$ branching fractions 
by different amounts, making the ratio the most sensitive observable. 
The impact on the ratio varies by model, but can be up to 20\% for two Higgs doublet or minimal supersymmetric models.

The ATLAS and CMS Collaborations have performed
searches for the decay $\PH\to\PZ\gamma\to\ell^+\ell^-\gamma$~\cite{atl-HZG,cms-HZG,Sirunyan:2018tbk,Aad:2020plj} at $\sqrt{s}=7$,
$8$, and $13\TeV$ in the $\Pe^+\Pe^-\gamma$ and $\mu^+\mu^-\gamma$ final states. 
The most stringent bound has been set by the ATLAS Collaboration using $\sqrt s = 13\TeV$ data corresponding to an integrated luminosity of $139\fbinv$. The observed (expected) upper limit at 95\% confidence level (\CL) on $\sigma(\Pp\Pp\to\PH)\mathcal{B}(\PH\to\PZ\gamma)$ relative to the SM is $3.6$ ($2.6$), assuming $m_\PH=125.09\GeV$.
The ATLAS experiment has reported evidence at the 3.2 standard deviation level for the decay $\PH\to\ell^+\ell^-\gamma$ with $m_{\ell^+\ell^-} < 30\GeV$ using both of the dilepton channels~\cite{atlas_llgrun2}.
The CMS Collaboration has also searched for the
$\PH\to\ell^+\ell^-\gamma$ process with $m_{\ell^+\ell^-} < 50\GeV$\, in the dimuon channel at $\sqrt{s}=8$~\cite{2016341} and
$13$~\cite{Sirunyan:2018tbk}$\TeV$.  

This paper describes a search for the decay $\PH\to \PZ\gamma$, where $\PZ\to\ell^+\ell^-$. The data sample corresponds to an integrated luminosity of \LumiT\fbinv of $\Pp\Pp$ collisions at $\sqrt s = 13 \TeV$ accumulated between 2016 and 2018. 
The region at small dilepton invariant mass, $m_{\ell^+\ell^-} < 50$ \GeV, is excluded from the analysis. It contains a contribution from an additional process, $\PH \to \gamma^* \gamma \to \ell^+ \ell^- \gamma$~\cite{Htollg-FB-Sun}.
The sensitivity of the analysis is enhanced by searching for Higgs boson production in a variety of mechanisms, including gluon-gluon fusion ($\Pg\Pg\PH$); vector boson fusion (VBF)
%, leading to an associated dijet system with high invariant mass
; and the associated production of a Higgs boson with either weak vector bosons (V$\PH$, where V = $\PZ$ or $\PW$) or top quark pairs ($\ttbar\PH$). The dominant backgrounds arise from Drell--Yan production in association with an initial-state photon~($\PZ/\gamma^{*}$+$\gamma$) and Drell--Yan production in association with jets, where a jet or additional lepton is misidentified as a photon ($\PZ/\gamma^{*}$+jets). 
After using a variety of discriminating variables to suppress background in the different production mechanisms, the signal is identified as a narrow resonant peak around $m_\PH$ in the distribution of the $\ell^+\ell^-\gamma$ invariant mass ($m_{\ell^{+}\ell^{-}\gamma}$).

The data sample is divided into eight mutually exclusive categories according to (i) the presence of an additional lepton produced by $\PZ(\to\ell^+\ell^-)$ or $\PW(\to\ell\nu)$ decay, indicating the possible associated production of a Higgs boson with $\PW$ or $\PZ$ bosons, or $\ttbar\PH$ production with a leptonic top quark decay; (ii) the value of a multivariate analysis (MVA) discriminant characterizing the kinematic properties of a dijet system together with the $\ell^+\ell^-\gamma$ candidate, indicating possible VBF production; and (iii) the value of an MVA discriminant characterizing the kinematic properties of the $\ell^+\ell^-\gamma$ system. A simultaneous maximum likelihood fit is performed to the $m_{\ell^+\ell^-\gamma}$ distribution in each category. 

This paper is organized as follows. The CMS detector and event reconstruction are described in Section~\ref{sec:cms}, and the data and simulated event samples are described in Section~\ref{sec:samples}. Section~\ref{sec:preselection} outlines the event selection, and Section~\ref{sec:class} discusses the event categorization using the MVA discriminants described above. The statistical procedure, including the modeling of signal and background shapes in the $m_{\ell^+\ell^-\gamma}$ distributions, is presented in Section~\ref{sec:modeling}. Systematic uncertainties are discussed in Section \ref{sec:systematics}. The final results obtained from the fits are discussed in Section~\ref{sec:results}, followed by a summary in Section~\ref{sec:summary}.


\chapter{Theory}\label{sec:theory}

\section{The Standard Model}

The SM is currently our best theoretical framework for understanding the nature of fundamental particles. 
It is rooted in the idea that particles exist as excitations of quantum fields. These fields are constructed so as to obey 
fundamental symmetries of nature, and quantum field theory describes both the particles of our universe and their interactions. The SM is not only 
elegant and extensive, but provides a wide variety of measurable observables for the experimentalist to probe. So far, many 
measurements have been made of the known elementary particles and their interactions, and the predictions of the SM have held up in each case. In 
this respect, it is a wildly successful theory. In other respects, it is obviously incomplete. It does not account for the gravitational force, dark matter, or dark energy, among other phenomena. 
Thus, there is value in testing the SM even more carefully with experiments 
like those at the LHC, in hopes of refining our understanding and potentially discovering new physics. 

\section{The Elementary Particles} 

\begin{figure}[htb]
	\begin{center}
	\includegraphics[width=0.65\textwidth]{fig/theory/Standard_Model_of_Elementary_Particles.png}
		\caption[Table of elementary particles in the SM. The leftmost three columns correspond to three generations of fermions, 
		the fourth column from the left shows the gauge bosons which mediate the elementary forces, and the rightmost column shows the Higgs boson. 
		For each particle, mass, charge, and spin are labeled, with mass values given as of 2019.]
		{Table~\cite{SMImage} of elementary particles in the SM. The leftmost three columns correspond to three generations of fermions, 
		the fourth column from the left shows the gauge bosons which mediate the fundamental forces, and the rightmost column shows the Higgs boson. 
		For each particle, mass, charge, and spin are labeled, with mass values given as of 2019.}
		\label{fig:SMParticles}
	\end{center}
\end{figure}

Figure \ref{fig:SMParticles} catalogs the elementary particles of the SM. These particles come in two broad types: fermions and bosons. Fermions are particles of half-integer spin that obey 
the Pauli exclusion principle. They are, therefore, responsible for the structure of matter. 
Each fermion has a corresponding antiparticle that has opposite electrical charge, but is otherwise identical. 
The fermions are subdivided into two types in three generations, the generations corresponding to mass.
Quarks are fermions that carry both fractional electric charge and color charge, so they participate in the electroweak and strong interactions. 
Leptons are fermions that carry integer electric charge and participate in the electroweak interaction.
In contrast to fermions, bosons have integer spin. The vector gauge bosons function as mediators of the fundamental forces of nature. 
The gluons mediate the strong force, the photon the electromagnetic force, and the W and Z bosons the weak force. 
Finally, the electrically neutral, scalar Higgs boson emerges due to electroweak symmetry breaking, described later in this section. The interaction of the Higgs 
with the other particles in the SM is responsible for particle masses. 
In the context of this thesis, which is a search for \hzg{}, the most relevant elementary particles are the Higgs boson, Z boson, leptons, and photon. 
Therefore, the remainder of this section will describe electroweak theory and the Higgs mechanism in more detail. Other aspects of the SM, such as quantum chromodynamics, 
are of importance in collider physics, but of less relevance for this specific search. Such topics are omitted in order to narrow the scope of the discussion.

\section{Electroweak Theory}
To motivate the Higgs mechanism and Higgs interactions, it is useful to first consider the electroweak interaction absent the Higgs. The electroweak piece of the 
SM Lagrangian can be written as 
\begin{align}
	& \mathcal{L}_{EW} = -\frac{1}{4}A_a^{\mu\nu}A^a_{\mu\nu} - \frac{1}{4}B^{\mu\nu}B_{\mu\nu} + i\sum_L \bar{L}\gamma^{\mu}D_{\mu,L}L + i\sum_R \bar{R}\gamma^{\mu}D_{\mu,R}R \label{eqn:LEWK}\\
	& A^a_{\mu\nu} = \partial_{\mu}A_{\nu}^{a} - \partial_{\nu}A_{\mu} + g\varepsilon_{abc}W^{b}_{\mu}W^{c}_{\nu} \\
	& B_{\mu\nu} = \partial_{\mu}B_{\nu} - \partial_{\nu}B_{\mu} \\
	& D_{\mu,L} = \partial_{\mu} + \frac{ig}{2}\tau\cdot A_{\mu} + \frac{ig'}{2}B_{\mu}Y \label{eqn:DL}\\
	& D_{\mu,R} = \partial_{\mu} + \frac{ig'}{2}B_{\mu}Y. \label{eqn:DR}
\end{align}
In the above equations, $A_{\mu\nu}^a$ represents an SU(2)$_L$ triplet of gauge fields and $B_{\mu\nu}$ represents a U(1) gauge field. The third and fourth terms in equation \ref{eqn:LEWK} 
describe the interactions of these gauge fields with the fermions, both quarks and leptons. Note that the interaction depends on the chirality of the fermions, where the left-handed 
fermions are denoted as $L$ and the right-handed as $R$. The right-handed fermions interact only with the U(1) field, while the left-handed fermions interact with the U(1) and SU(2)$_L$ fields. 
This is encoded by the derivative operators $D_{\mu,L}$ and $D_{\mu,R}$ defined in equations \ref{eqn:DL} and \ref{eqn:DR} in terms of the fields, Pauli matrices ($\tau$), hypercharge operator Y, and the couplings $g$ and $g'$. 

\section{Spontaneous Symmetry Breaking (Higgs Mechanism)}
One important property of the electroweak Lagrangian in equation \ref{eqn:LEWK} is that the bosons associated with the gauge fields are all massless. 
However, the direct observation of charged and neutral 
current interactions \cite{HASERT1973121,HASERT1973138} at CERN in 1973 implied that the W and Z boson must have relatively large masses in the range of 50--100\GeV. Indeed, the massive W and Z bosons were later discovered and measured \cite{UA1:1983crd,UA2:1983tsx} at the CERN Super Proton Synchroton in 1983. 
A solution to this shortcoming of the electroweak theory came in the form of the Higgs mechanism \cite{Englert:1964et,Higgs:1964ia,Higgs:1964pj},
which spontaneously breaks the SU(2)xU(1) gauge symmetry. One consequence of this is the addition of a real 
Higgs field accompanied by a massive Higgs boson. The W and Z boson masses arise naturally due to the electroweak symmetry breaking, 
while fermion masses are explained via additional Yukawa couplings to the Higgs boson. 
The discovery of the Higgs boson and subsequent measurements of its properties have provided experimental verification that the Higgs mechanism 
is a central piece of the Standard Model. 

To understand how the Higgs mechanism works, consider the introduction of a complex scalar field $\sf \Phi$, which 
transforms as a doublet under SU(2)$_L$.
\begin{equation}
    \label{eqn:higgsField}
    \Phi = 
    \begin{bmatrix}
        \phi^{+} \\ 
        \phi^{0}
    \end{bmatrix}
\end{equation}
Its contribution to the Lagrangian is given by:
\begin{equation}
    \mathcal{L_{H}} = (D^{\mu}_L\Phi)^{\dagger}(D_{\mu,L}\Phi) - V(\Phi)
    \label{LHiggs}
\end{equation}
where the Higgs potential takes the form
\begin{equation}
    V(\Phi) = \mu^{2}|\Phi^{\dagger}\Phi| + \lambda \Big(|\Phi^{\dagger}\Phi|\Big)^{2}.
    \label{VHiggs}
\end{equation}
Consider the case in which the parameters of the Higgs potential $\lambda$ and $\mu$ satisfy the conditions 
$\lambda > 0$ and $\mu^{2} < 0$. Then the shape of the potential is shown in Fig. \ref{fig:VHiggs} (right). There is no minimum of the 
potential at $\Phi = 0$. Rather, an infinite set of minima lie around a circle in the complex plane. Hence, it is said that $\Phi$ 
has a nonzero vacuum expectation value (VEV). The value of the VEV in terms of $\mu$ and $\lambda$ can be determined by explicitly
minimizing the potential:
\begin{align*}
    \frac{\partial}{\partial(\Phi^{\dagger}\Phi)}V(\Phi) &= 0 \\
    \mu^{2} + 2\lambda\Big(|\Phi^{\dagger}\Phi|\Big) &= 0 \\
    \mu^{2} + 2\lambda\Big[(\phi^{+})^{2} + (\phi^{0})^{2}\Big] &= 0 \numberthis
    \label{VHiggsMinimization}
\end{align*}
This can be minimized in many ways depending on individual values of $\phi^{+}$ and $\phi^{0}$ in the vacuum. By convention,
and without loss of generality, we choose the case in which $\phi^{+} = 0$. In this case we obtain the equation
\begin{equation}
    \phi^{0} = \sqrt{\frac{-\mu^{2}}{2\lambda}} = \frac{1}{\sqrt{2}}v
\end{equation}
where we have defined $v \equiv \sqrt{-\mu^{2}/\lambda}$.

\begin{figure}
	\begin{center}
	\includegraphics[width=0.75\textwidth]{fig/theory/spont_sym_breaking.jpg}
		\caption{Comparison of the shape of the potential without symmetry breaking and with spontaneous symmetry breaking.}
		\label{fig:VHiggs}
	\end{center}
\end{figure}

The existence of the Higgs VEV has profound implications. To see this, it is helpful to reparameterize the scalar doublet field
$\Phi$ as follows:
\begin{equation}
    \Phi = \frac{1}{\sqrt{2}}e^{i\frac{\tau^{a}}{2}\theta_{a}(x)}
    \begin{bmatrix}
        0 \\
        v + h(x)
    \end{bmatrix}
    \label{reparamHiggsField}
\end{equation}
As $\Phi$ is invariant under local SU(2)$_L$ gauge transformations, the prefactor may be rotated away. This is equivalent to setting 
$\theta(x) = 0$ in equation \ref{reparamHiggsField}. This choice of gauge is known as the unitary gauge, and leads to 
\begin{equation}
    \Phi = \frac{1}{\sqrt{2}}
    \begin{bmatrix}
        0 \\
        v + h(x)
    \end{bmatrix}
    \label{HiggsFieldUnitaryGauge}
\end{equation}
Given the above form of $\Phi$, we can now evaluate the Higgs Lagrangian (equation \ref{LHiggs}), starting with the kinetic term
$(D^{\mu}_L\Phi)^{\dagger}(D_{\mu,L}\Phi)$.
\begin{align*}
    (D^{\mu}_L\Phi)^{\dagger}(D_{\mu,L}\Phi) &= \Big| (\partial_{\mu} - \frac{ig}{2}\tau\cdot A_{\mu} - \frac{ig'}{2}B_{\mu}Y)\Phi\Big|^{2} \\
    &= \frac{1}{2}
    \Big| 
    \begin{bmatrix}
        \partial_{\mu} - \frac{i}{2}(gA_{\mu}^{3} + g'B_{\mu}) & -\frac{ig}{2}(A_{\mu}^{1} - iA_{\mu}^{2}) \\
            -\frac{ig}{2}(A_{\mu}^{1} + iA_{\mu}^{2}) & \partial_{\mu} + \frac{i}{2}(gA_{\mu}^{3} - g'B_{\mu})
    \end{bmatrix} 
    \begin{bmatrix}
        0 \\ 
        v + h(x)
    \end{bmatrix} 
    \Big|^{2} \\ &=  
    \frac{1}{2}\Big| 
    \begin{bmatrix}
        -\frac{ig}{2}(A_{\mu}^{1} - iA_{\mu}^{2})(v + h(x)) \\
        \partial_{\mu}h(x) + \frac{i}{2}(gA_{\mu}^{3} - g'B_{\mu})(v + h(x))
    \end{bmatrix}
    \Big|^{2} \\ &= 
    \frac{1}{2}\partial_{\mu}h(x)\partial^{\mu}h(x) + \frac{1}{8}(gA_{\mu}^{3} - g'B_{\mu})(gA^{\mu}_{3} - g'B^{\mu})(v + h(x))^{2} \\ &+ 
    \frac{g^{2}}{8}(A_{\mu}^{1} - iA_{\mu}^{2})(A_{\mu}^{1} + iA_{\mu}^{2})(v + h(x))^{2} \label{kinHiggsExpansion} \numberthis 
\end{align*}
With some foreknowledge of the result, we define the physical gauge fields and their masses.
\begin{align}
    W_{\mu}^{\pm} &= \frac{1}{\sqrt{2}}(A_{\mu}^{1} \mp iA_{\mu}^{2}) \; &m_{W} = \frac{gv}{2} \\
    Z_{\mu} &= \frac{1}{\sqrt{g^{2} + g'^{2}}}(gA_{\mu}^{3} - g'B_{\mu}) \; &m_{Z} = \sqrt{g^{2} + g'^{2}}\frac{v}{2} \\
    A_{\mu} &= \frac{1}{\sqrt{g^{2} + g'^{2}}}(g'A_{\mu}^{3} + gB_{\mu}) \; &m_{A} = 0
    \label{gaugeBosonsAndMasses}
\end{align}
Then the kinetic term of the Higgs Lagrangian can be recast as
\begin{align*}
    (D^{\mu}_L\Phi)^{\dagger}(D_{\mu,L}\Phi) &= \frac{1}{2}\partial_{\mu}h(x)\partial^{\mu}h(x) \\ 
    &+ \frac{1}{2}m_{Z}^{2}Z_{\mu}Z^{\mu} + m_{W}^{2}W_{\mu}^{+}W^{-\mu} \\
    &+ \frac{v}{4}(g^{2} + g'^{2})Z_{\mu}Z^{\mu}h + \frac{1}{8}(g^{2} + g'^{2})Z_{\mu}Z^{\mu}h^{2} \\
    &+ \frac{v}{4}g^{2}W_{\mu}^{+}W^{-\mu}h + \frac{1}{8}g^{2}W_{\mu}^{+}W^{-\mu}h^{2} \label{kinHiggsMasses}\numberthis
\end{align*}
Equation \ref{kinHiggsMasses} provides a great deal of information on the physical ramifications of the Higgs mechanism.
The first term is the kinetic term of the physical Higgs boson field. The second and third terms are the mass terms of the Z and W 
bosons, respectively. The fourth and fifth terms show the linear and quadratic couplings of the Z boson to the Higgs boson, respectively.
Finally, the sixth and seventh terms show the linear and quadratic couplings of the W boson to the Higgs boson, respectively. 
Given the form of equation \ref{HiggsFieldUnitaryGauge}, a similar expansion can be carried out on the Higgs potential
(equation \ref{VHiggs}). Here, the terms involving the physical Higgs boson are most interesting, so constant terms are dropped.
\begin{align*}
    V(\Phi) &= \frac{\mu^{2}}{2}(v+h)^{2} + \frac{\lambda}{4}((v+h)^{2})^{2} \\
    &\rightarrow \lambda v^{2}h^{2} + \lambda vh^{3} + \frac{\lambda}{4}h^{4} \label{expandedVHiggs} \numberthis
\end{align*}
The first term is a Higgs mass term, with $m_{H} = \sqrt{2\lambda v^{2}}$. The second and third terms describe the Higgs 
trilinear and quartic self-couplings, respectively.

The Lagrangian including the Higgs doublet $\Phi$ can be further extended to incorporate interactions between the Higgs and fermion fields. These interactions, along with the nonzero Higgs VEV, provide a mechanism to generate the fermion masses. The interactions take the 
form of Yukawa couplings:
\begin{equation}
    \mathcal{L}_{Yukawa} = Y_{ij}^{d}\bar{Q}^{i}_{L}\Phi d_{R}^{j} + Y_{ij}^{u}\bar{Q}^{i}_{L}\tilde{\Phi} u_{R}^{j} 
    + Y_{ij}^{e}\bar{L}^{i}_{L}\Phi e_{R}^{j} + h.c.,
    \label{LYukawa}
\end{equation}
where $\tilde{\Phi} \equiv i\tau_{2}\Phi^{*}$, and $u$, $d$, and $e$ represent up-type quarks, down-type quarks, and leptons, respectively. The constants $Y_{ij}^a$ denote the Yukawa couplings in each case, $Q^i$ is the set of SU(2) quark doublets, and $L^i$ is the set of SU(2) lepton doublets.
\begin{equation}
    \tilde{\Phi} \equiv i\tau_{2}\Phi^{*}
    \label{PhiDual}
\end{equation}
Plugging in the unitary gauge parameterization of equation \ref{HiggsFieldUnitaryGauge}, this evaluates to 
\begin{align}
    \mathcal{L}_{Yukawa} = \frac{Y_{ij}^{d}}{\sqrt{2}}\bar{d}_{L}^{i}(v + h)d_{R}^{j} 
    + \frac{Y_{ij}^{u}}{\sqrt{2}}\bar{u}_{L}^{i}(v+h)u_{R}^{j} + \frac{Y_{ij}^{e}}{\sqrt{2}}\bar{e}_{L}^{i}(v+h)e_{R}^{j}
    \label{expandedLYuk}
\end{align}
It is worth looking closely at the terms in equation \ref{expandedLYuk}. For a given fermion type, the first term in the parentheses
is a fermion mass term. The second term in the parentheses gives the coupling of the fermion to the Higgs boson. With this in mind, 
we observe that the fermion mass is given in terms of the couplings and the Higgs vev by 
\begin{equation}
    m_{f} = \frac{y_{f}v}{\sqrt{2}}
    \label{fermionMass}
\end{equation}
where $y_{f}$ is the relevant value taken from the Yukawa coupling matrix. In addition, we see that the strength of the fermion coupling
to the Higgs boson is given by $\frac{m_{f}}{v}$. Thus, fermions couple to the Higgs boson with strength directly proportional 
to their masses. This result has important consequences in the context of collider experiments, as it regulates the rates of
Higgs production and decay related to each fermion-Higgs interaction. 

%The existence of the Higgs VEV leads naturally to the masses of the W and Z bosons. This can be shown by squaring the covariant 
%derivative acting on the scalar doublet $\Phi$ and evaluating the result in the vacuum. Note that in the vacuum, all terms involving 
%the partial derivative $\partial_{\mu}$ will yield zero contribution. Therefore, keeping only the relevant terms, the squared covariant 
%derivative reduces to
%\begin{equation}
%    |D_{\mu}|^{2} \rightarrow (\frac{g}{2}A_{\mu}^{a}\tau^{a} +\frac{g'}{2}B_{\mu})(\frac{g}{2}A^{b\mu}\tau^{b} + \frac{g'}{2}B^{\mu})
%    \label{covDerivSquared}
%\end{equation}
%Evaluating this in the vacuum yields
%\begin{align}
%    \Delta\mathcal{L} &= \frac{1}{2}
%    \begin{bmatrix}0 & v
%    \end{bmatrix}
%    (\frac{g}{2}A_{\mu}^{a}\tau^{a} +\frac{g'}{2}B_{\mu})(\frac{g}{2}A^{b\mu}\tau^{b} + \frac{g'}{2}B^{\mu}) 
%    \begin{bmatrix}
%        0 \\ 
%        v
%    \end{bmatrix} \\ 
%    \Delta\mathcal{L} &=
%    \frac{1}{2}\frac{v^{2}}{4}[g^{2}(A_{\mu}^{1})^{2} g^{2}(A_{\mu}^{2})^{2} + (g'B_{\mu} - gA_{\mu}^{3})^{2}]
%\end{align}
%From this, we can identify the fields and masses for the positively and negatively charged W bosons and the neutral Z boson and photon. 
%These are as follows:


\section{Higgs Production}

At a hadron collider like the LHC, the Higgs boson can be produced via several different mechanisms, each mechanism occuring at a certain rate. The dominant production 
mechanism is gluon-gluon fusion (ggH), followed by vector boson fusion (VBF) production, which is about an order of magnitude rarer. Less common production mechanisms, but 
still relevant for a search at the LHC, are the associated production mechanisms WH, ZH, and t$\bar{t}$H. The cross sections for each Higgs production mechanism as a function of 
center of mass energy are shown in Fig. \ref{fig:higgs_prod}.

\begin{figure}
	\begin{center}
	\includegraphics[width=0.75\textwidth]{fig/theory/Plot_Escan_H125_new_sqrt.pdf}
	\caption{Cross sections for different Higgs boson production mechanisms as a function of center of mass energy.}
	\label{fig:higgs_prod}
	\end{center}
\end{figure}

\section{Higgs Decay}

The Higgs boson decays rapidly after it is produced at the LHC. It can decay into a variety of final states, each occuring at a certain rate. Tree-level Higgs decays occur at rates 
proportional to the square of the mass of the decay products. Decays to b$\bar{b}$, WW, $\tau^+\tau^-$, and ZZ, are among the most common and experimentally accessible at the LHC. Other Higgs decays 
occur at loop-level, including $\PH \to \Pg\Pg$ and \hzg. Figure \ref{fig:higgs_br} shows the branching fractions at $\sqrt{s}=13$\TeV for different decay channels as a function of Higgs boson mass. 
We see that the branching fraction of \hzg is suppressed relative to several other production modes, so a search for \hzg must rely on the clean $\ell^+\ell^-\gamma$ signature.

\begin{figure}
	\begin{center}
	\includegraphics[width=0.75\textwidth]{fig/theory/SMHiggsBR.YR4-rect.pdf}
		\caption{Branching fraction of \hzg decay as a function of Higgs boson mass.}
		\label{fig:higgs_br}
	\end{center}
\end{figure}

\section{Physics Beyond the Standard Model}

One of the goals of the LHC is to search for BSM physics. While the decay \hzg{} is predicted in the SM, searching for it and measuring its branching fraction can indirectly probe potential 
new physics phenomena. A few unique features of \hzg{} make it well-suited as a BSM probe. First, the decay is loop-induced, which means that the introduction of new particles will contribute 
to the loop, interfering and altering the final branching fraction. Secondly, the decay \hgg{} is a very similar loop-induced process that has been discovered and measured extensively 
by the CMS and ATLAS Collaborations [REFS]. This means any new particle content entering into the \hzg{} loop will also enter into the \hgg{} loop. However, in the case of \hzg{}, the
Z boson couples to loop particles according to the SU(2)xU(1) quantum number rather than just the electric charge, as is the case for \hgg{}. 
It follows that potential BSM physics scenarios can shift the 
branching fractions of \hzg{} and \hgg{} by differing amounts, making the ratio $\mathcal{B}(\PH\rightarrow\PZ\gamma)/\mathcal{B}(\PH\rightarrow\gamma\gamma)$ a sensitive observable. In the SM, 
$\mathcal{B}(\PH\rightarrow\PZ\gamma)/\mathcal{B}(\PH\rightarrow\gamma\gamma) = 0.69 \pm 0.04$. A measurement of a significant deviation from this value would be an indicator of possible BSM physics.
Below, we describe a few examples of BSM scenarios that can affect this ratio.

One way to consider BSM effects in the context of \hzg{} is to introduce new particles, like a W' boson, charged scalar, or a pair of charged leptons. 
This is the approach taken by Carena, Low, and Wagner~\cite{Zg_theory_decaywidth}. The W' possibility is motivated by the fact that the W boson loop is the dominant 
contribution in the SM. The W' can be defined by an SU(2) triplet and characterized by mass and Higgs coupling parameters. Figure \ref{fig:rzg_bsm} (left) shows contours of constant branching fraction 
enhancement in the \hzg{} and \hgg{} decay channels in the mass-coupling plane for the W' model. 
Depending on the parameter values, the level of enhancement can be up to double the SM value, and it differs 
significantly in the two channels. Similarly, we can imagine a new charged scalar or a pair of new charged leptons. The branching fraction enhancement contour plots in Fig. \ref{fig:rzg_bsm} show that for similar regions in the mass-coupling plane, \hzg{} can be enhanced, while \hgg{} is simultaneously diminished. 

Other relevant BSM physics models include extended Higgs sectors. Such models often contain charged Higgs bosons, which enter into the \hzg{} and \hgg{} loops and impact the branching fractions. 
In Ref. \cite{Zg_theory_extension}, Chiang and Yagyu catalog extended Higgs models of varying types: models with one singly-charged boson, those with one singly-charged and one doubly-charged boson, 
and those with two singly-charged bosons. In total, the authors consider thirteen models, including the two Higgs doublet model, the minimal SUSY model, and the Higgs triplet model. 
Figure \ref{fig:exthiggs} gives the ratio of decay widths $\Gamma(\PH\rightarrow\PZ\gamma)/\Gamma(\PH\rightarrow\gamma\gamma)$ as a function of the mass coefficient of the additional scalar field(s) for seven of the models. It is evident that, for mass scales accessible at the LHC, the ratio can be significantly shifted relative to the SM expectation. 


\begin{figure}[tb]
	\begin{center}
		\includegraphics[width=0.30\textwidth, height=.30\textwidth]{fig/theory/rzgamma_wprime.png}
		\includegraphics[width=0.30\textwidth, height=.30\textwidth]{fig/theory/rzgamma_scalar.png}
		\includegraphics[width=0.30\textwidth, height=.30\textwidth]{fig/theory/rzgamma_fermion.png}
		\caption[Contours of constant branching fraction enhancement relative to the SM for the W' (left), charged scalar (center), 
		and charged lepton (right) scenarios. 
		The solid lines represent contours of enhancement in $\mathcal{B}$(\hzg), while the dashed lines correspond to \hgg{}. The yellow boxed numbers label the enhancement for each 
		\hzg{} contour, and the unboxed numbers label the enhancement for each \hgg{} contour.]
		{Contours of constant branching fraction enhancement relative to the SM for the W' (left), charged scalar (center), 
		and charged lepton (right) scenarios~\cite{Zg_theory_decaywidth}. 
		The solid lines represent contours of enhancement in $\mathcal{B}$(\hzg), while the dashed lines correspond to \hgg{}. The yellow boxed numbers label the enhancement for each 
		\hzg{} contour, and the unboxed numbers label the enhancement for each \hgg{} contour.}
		\label{fig:rzg_bsm}
	\end{center}
\end{figure}

\begin{figure}[tb]
	\begin{center}
		\includegraphics[width=0.9\textwidth]{fig/theory/rzgamma_exthiggs.png}
		\caption[Ratio of decay widths $\Gamma(\PH\rightarrow\PZ\gamma)/\Gamma(\PH\rightarrow\gamma\gamma)$ as a function of the mass coefficient of the additional scalar field(s) for 
		seven BSM models with extended Higgs sectors. 
		The models plotted on the left correspond to one additional singly-charged boson, and those on the right to one additional singly-charged and one additional doubly-charged boson.]
		{Ratio of decay widths $\Gamma(\PH\rightarrow\PZ\gamma)/\Gamma(\PH\rightarrow\gamma\gamma)$ as a function of the mass coefficient of the additional scalar field(s) for 
		seven BSM models with extended Higgs sectors~\cite{Zg_theory_extension}. 
		The models plotted on the left correspond to one additional singly-charged boson, and those on the right to one additional singly-charged and one additional doubly-charged boson.}
		\label{fig:exthiggs}
	\end{center}
\end{figure}


\chapter{Experiment Description}\label{sec:experiment}

\section{The Large Hadron Collider}
The Large Hadron Collider (LHC) is a high energy proton-proton collider that serves several modern particle physics detector experiments. 
Built at CERN and straddling the French-Swiss border, and first turned on for operation in 2008, it is the most powerful particle collider in history. 
The LHC was designed to collide protons at a maximum center of mass energy of $\sqrt s = 14\TeV$ 
with a maximum instantaneous luminosity of $10^{34}cm^{-2}s^{-1}$. To achieve this level of performance, the 27 km tunnel 
originally constructed for the Large Electron-Positron Collider (LEP) was repurposed to separately 
accelerate two couterrotating proton beams. The separate acceleration is made possible by oppositely oriented 
magnetic dipole fields in the two rings. When the beams achieve the desired energy, they are collided at one of four interaction points. 

The protons collided in the LHC are first gathered, bunched, and accelerated by other parts of the CERN accelerator complex before injection into the LHC ring. 
First, protons are stripped off of hydrogen atoms by a duoplasmatron and accelerated to 50\MeV by a linear accelerator (LINAC). Next, the protons are further accelerated 
in a series of synchroton rings of increasing size, the Proton Synchroton Booster (PSB), Proton Synchroton (PS), and Super Proton Synchroton (SPS). The SPS brings the 
proton beam energy up to 450\GeV. At this point, the beam is injected to the two LHC rings to be accelerated up to the final collision energy. Acceleration in the main 
ring is achieved by a system of thousands of superconducting dipole magnets, along with hundreds of correcting quadrupole magnets. A liquid helium cooling system is 
used to maintain the magnets at cryogenic temperatures. A full diagram of the CERN accelerator complex, including the parts relevant for the LHC, is shown in Fig. \ref{fig:accelerator_complex}.

\begin{figure}
  \centering
   \includegraphics[width=0.9\textwidth]{fig/experiment/CCC-v2019-final-white.png}
	\caption{CERN accelerator complex.}
	\label{fig:accelerator_complex}
\end{figure}


\section{The Compact Muon Solenoid}
The CMS apparatus~\cite{CMS:2008xjf} is a multipurpose, nearly hermetic detector, designed to trigger on~\cite{CMS:2020cmk,CMS:2016ngn} and identify electrons, muons, photons, and (charged and neutral) hadrons~\cite{CMS:2015xaf,CMS:2018rym,CMS:2015myp,CMS:2014pgm}.
The central feature of the CMS apparatus is a superconducting solenoid of 6\unit{m} internal diameter, providing a magnetic field of $3.8$\unit{T}. Within the solenoid volume are a silicon pixel and strip tracker, a lead tungstate crystal electromagnetic calorimeter (ECAL), and a brass and scintillator hadron calorimeter (HCAL), each composed of a barrel and two endcap sections. The ECAL consists of 75\,848 lead tungstate crystals, which provide coverage in pseudorapidity $\abs{\eta} < 1.48 $ in a barrel region (EB) and $1.48 < \abs{\eta} < 3.0$ in two endcap regions (EE). Preshower detectors consisting of two planes of silicon sensors interleaved with a total of $3$ radiation lengths of lead are located in front of each EE detector. Forward calorimeters extend the pseudorapidity coverage provided by the barrel and endcap detectors. Muons are measured in gas-ionization detectors embedded in the steel flux-return yoke outside the solenoid. A more detailed description of the detector components and operation is provided below. 
%A more detailed description of the CMS detector, together with a definition of the coordinate system used and the relevant kinematic variables, can be found in Ref.~\cite{CMS:2008xjf}.

\begin{figure}
  \centering
   \includegraphics[width=0.9\textwidth]{fig/experiment/detector/cms_about_detector.png}
	\caption{The CMS detector apparatus.}
	\label{fig:detector}
\end{figure}


\subsection{Superconducting Magnet}
The function of the superconducting magnet within the CMS detector is to bend the trajectories of charged particles. This is crucial for particle identification and momentum measurement. 
Designed to provide a maximum magnetic field of $4$\unit{T}, its operating strength during pp collision runs is set to $3.8$\unit{T}. The bulk of the magnet is composed of NbTi, cooled by 
liquid helium to 4.5K, which is below the critical temperature allowing for superconductivity. The magnet has a length of 12.5m, diameter of 6.3m, and mass of 220 tons. The solenoid encloses 
several detector components, including the tracker and the majority of the calorimeters. A steel flux-return yoke is built around the solenoid, and is composed of 5 wheels and two endcaps weighing 
a total of roughly 10,000 tons. 

\subsection{Inner Tracking System}
The inner tracking system is designed to reconstruct charged particle trajectories and vertices arising from particle decays. Comprised of a silicon pixel detector and silicon strip tracker, 
it covers the pseudorapidity region $|\eta| < 2.5$ and is designed for efficient and precise measurement of charged particles with \pt about about 1\GeV. 
At the LHC design luminosity for pp collisions, each bunch crossing leads to about 1,000 hits in the inner tracking system. As such, the inner tracking system was designed to be 
maximally radiation tolerant while maintaining physics performance. 

The silicon pixel detector covers the inner region of $r < 10 cm$. It is composed of $100 \times 150 \mu m^{2}$ pixels arranged in three barrel layers and two endcap disks. Due to the extreme 
radiation environment, the innermost layer was designed to be replaced after at least two years of LHC operation. 
In response to LHC running conditions and detector degradation, a replacement and upgrade of the full pixel detector was made during the LHC extended year-end technical stop in 2016/2017. 
The silicon strip tracker covers the region $20 < r < 116 cm$. It is divided into three subsystems: the Tracker Inner Barrel (TIB), Tracker Inner Disks (TID), and Tracker Outer Barrel (TOB). 
The strip thickness is $320 \mu m$ ($500 \mu m$) in the TIB/TID (TOB). A schematic diagram of the CMS inner tracking system in the r-z plane is shown in Fig. \ref{fig:cms_tracker}.

\begin{figure}
  \centering
   \includegraphics[width=0.9\textwidth]{fig/experiment/detector/cms_tracker.png}
	\caption{Cross section of the CMS inner tracking system in the r-z plane.}
	\label{fig:cms_tracker}
\end{figure}


\subsection{Electromagnetic Calorimeter (ECAL)}
The electromagnetic calorimeter (ECAL) is designed to identify and measure electromagnetic particles by inducing and characterizing electromagnetic showers. 
It is made of 61,200 lead tungstate ($PbWO_4$) crystals in the barrel (EB), which covers $|\eta| < 1.479$, plus 7,324 crystals in each endcap (EE), which fall in the range $1.479 < |\eta| < 3.0$. A preshower system lies in front of the EE. The preshower is a lead and silicon sampling calorimeter designed to aid in the identification of neutral pions and to improve the spatial resolution in the 
endcap region. 
The crystal length corresponds to 25.8 (24.7) radiation lengths ($\chi^{0}$) in the EB (EE). When electromagnetic particles, such as 
electrons and photons, encounter the ECAL, electromagnetic showers are induced. Light is emitted by the ECAL crystals proportional to the energy of constituent particles in the shower, allowing 
for reconstruction of the full shower energy. An array of fast and radiation tolerant photodetectors detects the emitted light, and this information is saved off detector. 
For this purpose, the EB uses Hamatsu avalanche photodiodes, while the EE uses vacuum phototriodes. 



\subsection{Hadronic Calorimeter (HCAL)}
Measurements of hadrons, jets, and missing transverse momentum are important to the analysis goals of the CMS physics program. The hadronic calorimeter (HCAL) is instrumental in such 
measurements. In a nutshell, the HCAL is primarily a sampling calorimeter broken into several subdetectors. The barrel (HB) covers the range $|\eta| < 1.3$, the endcaps (HE) cover the range $1.3 < |\eta| < 3$, the forward calorimeter (HF) covers the range $3 < |\eta| < 5$, and an outer barrel calorimeter (HO) covers $|\eta| < 1.3$ outside of the solenoid magnet. 

Layers of brass absorber are interleaved with plastic scintillator in the HB and HE, with a total absorber thickness of 5.82--10.6 interaction lengths in the HB and about 10 interaction lengths in the HE. Scintillation light emitted by the active material is measured and transmitted off-detector using hybrid photodiodes.
The design of the HF was motivated by the extreme particle fluxes in the very forward region. The HF is made up of steel plate absorbers, in which quartz fibers are inserted to serve as the active material. Cherenkov light is produced when shower particles above the Cherenkov threshold 
pass through the fibers, and a calculable fraction of the light is captured by the fibers. The function of the HO is to recover the energy in showers that leak through the HB, which is crucial for the accurate measurement of missing transverse momentum. In the HO, the solenoid coil itself, as well as a thick iron tail catcher, function 
as absorbers. This extends the effective HCAL depth to 11.8 interaction lengths everywhere except at the barrel-endcap boundary. 

\subsection{Muon Detectors}
The CMS muon detector system is comprised of a set of gas ionization detectors embedded in the flux-return yoke outside the solenoid magnet. 
These detectors cover the pseudorapidity range $|\eta| < 2.4$. The positioning of the detectors outside the magnet takes advantage of the fact that muons deposit minimal energy in the detector 
materials, such as the calorimeters. Most other types of high energy particles will have deposited their energy before reaching the muon system, so using the combination of muon system and tracker information, CMS is able to reconstruct muons with excellent momentum resolution. The gas ionization detectors in the muon system come in three types: drift tubes (DTs), cathode strip chambers (CSCs), and 
resistive plate chambers (RPCs). The DTs cover the barrel region ($|\eta| < 1.2$) and are arranged in four stations, each with 60--70 drift chambers containing a gas mixture of 85\% Ar and 15\% CO$_2$.
The CSCs are multiwire proportional chambers composed of anode wire planes interleaved with cathode panels. Muons passing through the CSCs ionize a gas mixture of 40\% Ar, 50\% CO$_2$, and 10\% CF$_4$, 
and the electrons flow to the anodes, yielding a detectable avalanche of charge. Covering the region $|\eta| < 2.1$, the RPCs are composed of oppositely charged parallel plates enclosing a gas mixture 
of primarily $\mathrm{C}_2\mathrm{H}_2\mathrm{F}_4$. One advantage of the RPCs is their good time resolution, which is less than the LHC bunch spacing time of 25ns. This makes them useful for fast muon triggering that matches
muon tracks to the relevant bunch crossing. Figure \ref{fig:muon_system} shows a cross sectional view of the CMS detector highlighting the muon system components. 

\begin{figure}
  \centering
   \includegraphics[width=0.9\textwidth]{fig/experiment/detector/muon_sys_r-z.png}
	\caption{Diagram of CMS detector in the r-z plane showing the components of the muon system.}
	\label{fig:muon_system}
\end{figure}

\section{Trigger System}
At the LHC, the proton beam crossing interval is 25 ns, which corresponds to a pp collision rate of 40 MHz. Recording the full set of detector information for each collision is unfeasible, as it 
would lead to far too much data to save to disk. As a result, the CMS experiment employs a two-tiered trigger system designed to preserve information of physics interest while reducing the 
stored event rate to a more manageable 100 Hz. The first level of the trigger (L1) is a hardware trigger, which reduces the event rate from 40 MHz to about 100 kHz. The second layer is the 
high level trigger (HLT), a processor farm using software optimized for fast processing. 

The L1 trigger is implemented using FPGAs and ASICs, and has local, regional, and global components. First, local patterns in the calorimeters, track segments, and hit patterns in the muon chambers 
are used by the Trigger Primitive Generators (TPGs). The TPGs rank and sort primitive objects corresponding to particle candidates based on energy, momentum, and reconstruction quality. Information
from the TPGs is taken as input by regional triggers for the calorimeters and muon systems. These regional triggers further determine candidate physics objects, as well as energy sums and isolation
information. Finally, this information is fed to a set of global triggers, which rank trigger objects across the entire detector. The final global trigger must decide whether to accept or 
reject an event based on information input from the global calorimeter and muon trigger systems. Algorithms used in this decision can be based on single object \pT thresholds and multiplicity-related 
thresholds, such as the presence of multiple jets, among other criteria. Events passing the L1 trigger are read out and passed to the HLT for further processing. 

In contrast to the L1 trigger, the HLT incorporates the full physics object reconstruction based on the full precision of the detector. This allows the HLT to accept and reject events with algorithms 
of similar quality to those used by offline analyses. Over 13,000 CPU cores are dedicated to HLT processing. For speed and efficiency, reconstruction and filtering algorithms are applied in increasing order of complexity. If a filter sequence fails, the rest of the reconstruction is skipped. Additionally, processing is done regionally based on the L1 candidates and relevant detector components 
passed as input to the HLT. Events that pass the HLT are assigned to relevant data streams based on their physics content. For this analysis, the relevant streams are the Double Muon and 
Double Electron streams, triggered by events with two muons or two electrons passing a set of minimum \pT requirements, among other cuts. 

\section{Object Reconstruction}
The global event reconstruction (also called particle-flow (PF) event reconstruction~\cite{CMS:2017yfk}) aims to reconstruct and identify each individual particle in an event, with an optimized combination of all subdetector information. In this process, the identification of the particle type (photon, electron, muon, charged hadron, neutral hadron) plays an important role in the determination of the particle direction and energy.

\begin{figure}
  \centering
   \includegraphics[width=0.9\textwidth]{fig/experiment/reconstruction/cms_detector.png}
	\caption{Schematic of different particles interacting with the CMS detector.}
\end{figure}

\subsection{Photon and Electron Reconstruction}
Photons and electrons interact with the material in the ECAL, depositing the majority of their energy before reaching the HCAL. As they interact, photons convert into 
electron-positron pairs and electrons radiate bremsstrahlung photons. This leads to the formation of an electromagnetic shower. Because of this, the original particle energy 
is split into multiple energy deposits, which must be combined to reconstruct the original energy. Deposits in individual ECAL crystals are combined into clusters, 
and these clusters are in turn combined into superclusters. Two algorithms are used to generate superclusters: the so-called "mustache" algorithm, and the so-called "refined" algorithm.
The mustache algorithm defines a seed cluster with energy above a certain threshold, and then combines it with other clusters within a region of the eta-phi plane centered at the 
seed location. The refined algorithm uses the mustache superclusters as well as tracking information to extrapolate bremsstrahlung and conversion tracks to decide whether a cluster should belong 
to the supercluster. 

The distinction between photons and electrons is made using tracking information, where photons are associated with no tracks and electrons with tracks. The Gaussian Sum Filter (GSF) [REF] track 
fitting algorithm is used to identify and characterize tracks that might be associated with an electron. It first begins with a hit pattern in the tracker, which is used as a seed. This seed 
can either be tracker-driven, coming from the collection of generic tracks tested for mutual compatibility, or it can be ECAL-driven, where a mustache supercluster is compared in location with 
a collection of tracker hit patterns to determine if the supercluster is consistent with the trajectory indicated by the track. Electron seeds are then converted into reconstructed electron tracks. 
In the absence of any GSF electron tracks, a photon candidate is obtained. Additional separation between photons and electrons is obtained through further selection requirements. 
The measured energy resolution for electrons produced in $\PZ$ boson decays in  $\Pp\Pp$ collision data ranges from $2$--$5$\%, depending on electron pseudorapidity and energy loss through bremsstrahlung in the detector material~\cite{CMS:2020uim}.

\subsection{Muon Reconstruction}
Muon reconstruction utilizes information from the muon detectors and the tracker. First, detector hits in the CSCs, DTs, and RPCs are used to build standalone tracks using a Kalman-filter technique.
Subsequently, these standalone muon tracks are combined with tracker information via two algorithms. So-called "tracker muons" are reconstructed using an "inside-out" algorithm, which starts from 
tracker tracks and matches them to DT or CSC segments. So-called "global muons" are reconstructed with an "outside-in" approach which starts from standalone muon tracks and matches them 
to tracker tracks using a Kalman-filter technique. In the case where both algorithms reconstruct a muon sharing the same tracker tack, the two outputs are merged into a single muon candidate.
In general, the tracker muon algorithm is more efficient in the region of low muon \pT, while the global algorithm is efficient at high \pT. 

The energy of muons is obtained from the corresponding track momentum. Matching muons to tracks measured in the silicon tracker results in a \pt resolution, for muons with \pt up to $100$\GeV, of $1$\% in the barrel and $3$\% in the endcaps. The \pt resolution in the barrel is better than $7$\% for muons with \pt up to $1$\TeV~\cite{CMS:2018rym}.


\subsection{Hadrons}
Charged hadrons are identified as charged particle tracks that are neither identified as electrons nor as muons. Neutral hadrons are identified as HCAL energy clusters not linked to any charged hadron trajectory, or as a combined ECAL and HCAL energy excess with respect to the expected charged hadron energy deposit.

\subsection{Jets}
For each event, hadronic jets are reconstructed as clusters of PF particles, including photons, electrons, muons, and hadrons. For clustering, the infrared and collinear-safe anti-\kt algorithm~\cite{Cacciari:2008gp, Cacciari:2011ma} is used. The goal of jet clustering algorithms is to combine PF particles together into a single jet object by merging them together based on their relative geometric distances and transverse momenta (\kt). In the case of the anti-\kt algorithm, clustering proceeds from smallest geometric distance to largest, weighted by squared inverse \kt. This inverse weighting 
preferentially combines PF particles with higher momenta before lower momenta, for a given distance. The advantage of this approach is that low momentum particles are more likely to be clustered 
with high momentum counterparts, rather than with other low momentum particles. It follows that these low momentum particles will not significantly modify the jet shape, whereas the more important, high-momentum particles, will. A precise definition of the anti-\kt algorithm is defined in equation \ref{eq:antikt}. 

\begin{equation}
	d_{ij} = min(k_{T,i}^{2p}, k_{T,j}^{2p})\frac{(\eta_i - \eta_j)^2 + (\phi_i - \phi_j)^2}{R^2},\;\; d_i = k_{T,i}^{2p}
	\label{eq:antikt}
\end{equation}

In equation \ref{eq:antikt}, the indices i and j refer to two PF particles, the parameter $p$ is taken to be -1, and the radius parameter $R$ is parameter that can be tuned. For standard jets in CMS, as used in this analysis, $R$ is set to 0.4. Particles are clustered from smallest $d_{ij}$ to largest until there is no $d_{ij}$ smaller than $k_{ti}^{-2}$, at which point the final jet is defined. A visual representation of the anti-\kt algorithm is shown in Fig. \ref{fig:anti_kt_diagram}.

\begin{figure}
  \centering
   \includegraphics[width=0.9\textwidth]{fig/experiment/reconstruction/jet_clustering.png}
	\caption{An example of the first few steps of the anti-\kt jet clustering algorithm.}
	\label{fig:anti_kt_diagram}
\end{figure}

Jet momentum is determined as the vectorial sum of all particle momenta in the jet, and is found from simulation to be, on average, within $5$--$10$\% of the true momentum over the entire \pt spectrum and detector acceptance. Additional $\Pp\Pp$ interactions within the same or nearby bunch crossings (pileup) can contribute additional tracks and calorimetric energy depositions to the jet momentum. To mitigate this effect, charged particles identified to be originating from pileup vertices are discarded and an offset correction is applied to correct for remaining contributions~\cite{CMS:2020ebo}. The jet energy resolution typically amounts to $15$--$20$\% at $30$\GeV, $10$\% at $100$\GeV, and $5$\% at $1$\TeV~\cite{CMS:2016lmd}.

\begin{figure}
  \centering
   \includegraphics[width=0.9\textwidth]{fig/experiment/reconstruction/jet_energy_resolution.png}
	\caption{Jet energy resolution as a function of simulated jet \pT in the barrel (left) and in the endcap (right) regions. Calo refers to the sum of ECAL and HCAL energy deposits, 
	while PF corresponds to jets clustered from PF particles using the anti-\kt algorithm.}
	\label{fig:anti_kt_diagram}
\end{figure}



\chapter{Analysis Overview}\label{sec:analysis_overview}

\section{History of the \hzg{} Search}
Both the CMS and ATLAS collaborations have undertaken the search for \hzg{} since Run 1 of the LHC. 
Before the CMS Run 2 analysis, which is the focus of this thesis, each collaboration published results using 
Run 1 data at $\sqrt s = 7$ and 8\TeV~\cite{cms-HZG,atl-HZG}, CMS published a result with 2016 data at $\sqrt s = 13\TeV$~\cite{Sirunyan:2018tbk}, and 
ATLAS published a full Run 2 result at $\sqrt s = 13\TeV$~\cite{Aad:2020plj}. As such, the present analysis represents the continuation 
of a broader effort, so we will give a brief history of these prior searches to put our work in context. While 
some of the analysis strategies of past searches inform our current strategy, there are also many differences 
and innovations in our analysis. We will emphasize the ways in which our analysis 
overlaps and differs with previous approaches, and will show that our innovations have significantly 
improved the expected sensitivity and statistical robustness of the search. 

\subsection{The Run 1 Searches}
In 2014, ATLAS published its Run 1 results~\cite{atl-HZG} for the search for \hzg{} where $\mathrm{Z}\to\ell^+\ell^-$ with $\ell=\mathrm{e}$ or $\mu$. The baseline strategy was a standard dilepton 
plus photon selection, with $m_{\ell^+\ell^-}$ near the Z boson mass and $\ell^+\ell^-\gamma$ invariant mass ($m_{\ell^{+}\ell^{-}\gamma}$) between 115 and 170\GeV. The resolution of $m_{\ell^{+}\ell^{-}\gamma}$ was improved using a kinematic fit procedure, accounting for the true Z boson line shape. Events were categorized based on lepton flavor; the pseudorapidity difference between the photon and Z boson; and the variable $p_{\mathrm{T}}^{t}$, defined as $\abs{\vec{p}_\mathrm{T}^{\,\PZ\gamma}\times\hat{t}}$, where $\hat{t}=(\vec{p}_\mathrm{T}^{\,Z}-\vec{p}_\mathrm{T}^{\,\gamma})/\abs{\vec{p}_\mathrm{T}^{\,Z}-\vec{p}_\mathrm{T}^{\,\gamma}}$~\cite{Ackerstaffetal.1998,VESTERINEN2009432}, the $\pt$ of the $\PZ\gamma$ system that is perpendicular to the difference of the three-momenta of the $\PZ$ boson and the photon, a quantity that is strongly correlated with the \pt of the $\lplm\gamma$ system. Limits on the signal strength relative to the SM prediction, shown in Fig. \ref{fig:run1_limits} (left), were obtained from a simultaneous fit to the $m_{\ell^{+}\ell^{-}\gamma}$ distributions in all categories.

The CMS Run 1 analysis~\cite{cms-HZG} was published in 2015 for the $\mpmm\gamma$ and $\epem\gamma$ final states. It employed a standard dilepton plus photon selection, but the $m_{\ell^+\ell^-\gamma}$ range was 100--190\GeV, which includes a kinematic turn-on around 110--115\GeV, mainly driven by the photon \pt selection. The choice to model this turn-on increased the complexity of the fit relative to the ATLAS analysis above, but was made in order to minimize potential bias. Events were categorized based on flavor; presence of a dijet system; lepton and photon pseudorapidity; and the photon $\mathrm{R}_9$, which is the energy sum of the 3x3 ECAL crystal array centered on the most energetic crystal divided by the supercluster energy, and is associated with the quality of the photon. The CMS Run 1 limits on the signal strength are shown in Fig. \ref{fig:run1_limits} (right). 

\begin{figure}[tb]
  \centering
   \includegraphics[width=0.45\textwidth,height=0.33\textwidth]{fig/overview/atl_run1_lim.png}
   \includegraphics[width=0.45\textwidth,height=0.33\textwidth]{fig/overview/cms_run1_lim.png}
	\caption
	[ATLAS (left) and CMS (right) Run 1 limit results as a function of $m_\PH$.]
	{ATLAS (left~\cite{atl-HZG}) and CMS (right~\cite{cms-HZG}) Run 1 limit results as a function of $m_\PH$.}
	\label{fig:run1_limits}
\end{figure}


\subsection{The Previous Run 2 Searches}
The first CMS \hzg{} results at $\sqrt s = 13\TeV$~\cite{Sirunyan:2018tbk} were published in 2018 using the 2016 data set, corresponding to an integrated luminosity of 35.9\fbinv. The search strategy was largely similar to the CMS Run 1 analysis, with a few main differences. First, a boosted category was introduced to the analysis, defined by $\pt^{\lplm\gamma}>60\GeV$. This category targeted events with a Higgs boson recoiling against a jet. Second, the range of $m_{\lplm\gamma}$ was restricted to 115--170\GeV in order to avoid fitting the kinematic turn-on. The decision of whether or not to fit the turn-on became critically important in the full CMS Run 2 analysis, and will be discussed in more detail later in this thesis. Limits from the CMS 2016 search are shown in Fig. \ref{fig:run2_prev_results} (left). 

The ATLAS full Run 2 search~\cite{Aad:2020plj} was published in 2020, with a largely similar strategy to the Run 1 analysis. As in Run 1, a kinematic fit was used to improve the $m_{\lplm\gamma}$ resolution and the variable $p_{\mathrm{T}}^{t}$ was used for categorization. However, there were a few major differences introduced in the Run 2 analysis. The fit range was changed to $105\GeV < m_{\lplm\gamma} < 160\GeV$ and some aspects of the categorization strategy were updated. For categorization, the most significant change was the use of a boosted decision tree (BDT) trained to separate VBF Higgs production events from other Higgs production mechanisms and backgrounds. As will be discussed later in this thesis, the use of BDTs for categorization can significantly improve the sensitivity of searches for \hzg{}. Figure \ref{fig:run2_prev_results} (right) shows a summary of the fit results for the ATLAS Run 2 search. An excess in data was observed, corresponding to $2.2$ standard deviations for $m_{\PH}=125.09\GeV$.

\begin{figure}[tb]
  \centering
   \includegraphics[width=0.45\textwidth,height=0.33\textwidth]{fig/overview/cms_2016_lim.png}
   \includegraphics[width=0.45\textwidth,height=0.33\textwidth]{fig/overview/atlas_run2_fit.png}
	\caption
	[Left: CMS 2016 limit results as a function of $m_\PH$. Right: ATLAS Run 2 signal plus background fit, for $m_{\PH}=125.09\GeV$, weighted by ln(1+$\mathrm{S}_{68}/\mathrm{B}_{68})$, where $\mathrm{S}_{68}$ and $\mathrm{B}_{68}$ are the expected signal and background events in an $m_{\lplm\gamma}$ window containing 68\% of the expected signal yield.]
	{Left: CMS 2016 limit results~\cite{Sirunyan:2018tbk} as a function of $m_\PH$. Right: ATLAS Run 2 signal plus background fit~\cite{Aad:2020plj}, for $m_{\PH}=125.09\GeV$, weighted by ln(1+$\mathrm{S}_{68}/\mathrm{B}_{68})$, where $\mathrm{S}_{68}$ and $\mathrm{B}_{68}$ are the expected signal and background events in an $m_{\lplm\gamma}$ window containing 68\% of the expected signal yield.}
	\label{fig:run2_prev_results}
\end{figure}

\section{The CMS Run 2 Strategy}


Before discussing the full details of the analysis procedure and results, it is worth summarizing 
the broader strategy taken in our search for \hzg. As two prior CMS results have 
been published with Run 1 data and 2016 data, we will emphasize the ways in which our analysis
overlaps and differs from these previous approaches. Very broadly, the current search is 
similar to the previous analyses in trigger, object, and basic event selection. 
However, it is significantly more advanced in three body mass reconstruction, 
event categorization, and background modeling. We will show that the innovations in these 
areas have significantly improved the expected sensitivity and statistical robustness of the search 
with respect to the past CMS analyses.

Chapter 5 provides a detailed description of the data and Monte Carlo simulation 
used in our analysis. Standard dimuon and dielectron trigger streams are used for 2016, 2017, 
and 2018 LHC data. The full dataset corresponds to an integrated luminosity of 
137 $fb^{-1}$. In other words, we use the full CMS Run 2 dataset at 13 TeV center of 
mass energy. Simulated signal samples are used to determine the expected signal yields 
and three body mass shape. Simulated background samples are used for MVA training 
and category optimization. However, the background shape and normalization in the final result
is determined by fitting the data and does not rely on any simulation.

Chapter 6 describes the basic object and event selection used in the analysis, and Chapter 7 
describes further selection and categorization using MVAs. As mentioned, the basic 
selection requirements are fairly similar to previous CMS analyses. Muons are selected with 
a loose cut-based ID, while electrons and photons are selected with loose MVA IDs. Loose ID
requirements are chosen in order to maximize signal efficiency. Background is then suppressed 
through a combination of basic cuts and MVA methods. Basic kinematic cuts on isolation, mass, 
and photon energy variables are able to significantly reduce backgrounds from initial and final 
state radiation, while the MVA methods are able to strongly disciminate against backgrounds from 
jets misreconstructed as photons. Finally, a kinematic fit procedure in the dilepton mass 
is able to significantly improve the signal mass resolution. The kinematic fit is an 
innovation of the current analysis and contributes to its improved sensitivity.

Chapter 8 details the approach to signal and background modeling of the three body 
mass spectrum. As in previous CMS \hzg searches, the 
signal shape is determined via an analytic fit to simulation. A resonant background 
contribution from \hmumu is modeled similarly. The nonresonant 
background contribution is taken from a fit to data in the range of 105 to 170 GeV. This 
background model includes both the turn-on arising from the real Z boson peak as well as 
the falling spectrum at higher mass. We note that the turn-on was fit in the Run 1 analysis as 
well, but was dropped in the 2016 analysis. A more thorough discussion of the merits of fitting 
the turn-on will be described in chapter 8.

Chapter 9 describes the systematic uncertainties relevant for the analysis, and Chapter 10
gives a basic overview of the statistics used to arrive at the final results. It is 
worth noting that given the current integrated luminosity, the analysis is dominated by 
statistical uncertainty. Chapter 11 gives the full set of results, including best fit 
signal strength, limits, and comparisons with \hgg. 




\chapter{Data and Simulated Samples}\label{sec:data}

The data sample used for the CMS Run 2 \hzg{} search corresponds to a total integrated luminosity of \LumiT\fbinv and was collected over a data-taking period spanning three years: \Lumia\fbinv in 2016, \Lumib\fbinv in 2017, and \Lumic\fbinv in 2018~\cite{CMS-LUM-17-003,LUM-17-004,LUM-18-002}. 
To be considered in the analysis, events must satisfy the HLT requirements for at least one of the dielectron or dimuon triggers.
The dielectron trigger requires a leading (subleading) electron with
$\pt > 23\,(12)\GeV$, while the dimuon trigger requires a leading (subleading) muon with $\pt > 17\,(8)\GeV$.
The corresponding CMS data streams are listed in Table \ref{tab:data_samples}. All data samples use the CMS MINIAOD data format.
To ensure data quality, luminosity masks are applied based on recommendations from the Physics Performance and Dataset group.

\begin{table}[tb]
  \begin{center}
	  \caption{Summary of MINIAOD data samples used.}
    \begin{tabular}{|l|l|}
      \hline
	    \textbf{$\Pe^+\Pe^-\PGg$ final state} & \textbf{$\PGm^+\PGm^+\PGg$ final state}	                     \\\hline 
      /DoubleEG/Run2016B-17Jul2018\_ver2-v1 	&  /DoubleMuon/Run2016B-17Jul2018\_ver2-v1   \\
      /DoubleEG/Run2016C-17Jul2018-v1 		&  /DoubleMuon/Run2016C-17Jul2018-v1         \\
      /DoubleEG/Run2016D-17Jul2018-v1 		&  /DoubleMuon/Run2016D-17Jul2018-v1         \\
      /DoubleEG/Run2016E-17Jul2018-v1	        &  /DoubleMuon/Run2016E-17Jul2018-v1         \\       
      /DoubleEG/Run2016F-17Jul2018-v1	        &  /DoubleMuon/Run2016F-17Jul2018-v1         \\       
      /DoubleEG/Run2016G-17Jul2018-v1	        &  /DoubleMuon/Run2016G-17Jul2018-v1         \\
      /DoubleEG/Run2016H-17Jul2018-v1	        &  /DoubleMuon/Run2016H-17Jul2018-v1         \\
      /DoubleEG/Run2016H-17Jul2018-v1	        &  /DoubleMuon/Run2016H-17Jul2018-v1         \\
      /DoubleEG/Run2017B-31Mar2018-v1	        &  /DoubleMuon/Run2017B-31Mar2018-v1         \\
      /DoubleEG/Run2017C-31Mar2018-v1	        &  /DoubleMuon/Run2017C-31Mar2018-v1         \\
      /DoubleEG/Run2017D-31Mar2018-v1	        &  /DoubleMuon/Run2017D-31Mar2018-v1         \\
      /DoubleEG/Run2017E-31Mar2018-v1	        &  /DoubleMuon/Run2017E-31Mar2018-v1         \\       
      /DoubleEG/Run2017F-31Mar2018-v1	        &  /DoubleMuon/Run2017F-31Mar2018-v1         \\           
      /EGamma/Run2018A-17Sep2018-v2		&  /DoubleMuon/Run2018A-17Sep2018-v2         \\  
      /EGamma/Run2018B-17Sep2018-v1		&  /DoubleMuon/Run2018B-17Sep2018-v1         \\
      /EGamma/Run2018C-17Sep2018-v1		&  /DoubleMuon/Run2018C-17Sep2018-v1         \\
      /EGamma/Run2018D-22Jan2019-v2		&  /DoubleMuon/Run2018D-PromptReco-v2        \\\hline       
    \end{tabular}
    \label{tab:data_samples}
  \end{center}
\end{table}

Signal samples for $\Pg\Pg\PH$, VBF, $\mathrm{V}\PH$, and $\ttbar\PH$ production, with 
$\PH\to\PZ\gamma$ and $\PZ\to\ell^+\ell^-$ ($\ell = \Pe$, $\mu$, or $\tau$),
are generated at next-to-leading order (NLO) using \POWHEG v2.0~\cite{cite:powheg1,cite:powheg2}.
These samples are produced for $m_\PH$ of $120$, $125$, and $130$\GeV. 
The dominant backgrounds, $\PZ/\gamma^{*}(\rightarrow \ell^+\ell^-)$+$\gamma$ and $\PZ/\gamma^{*}(\rightarrow \ell^+\ell^-)$+jets,
are generated at NLO using the \MGvATNLO v2.6.0 (v2.6.1) 
generator~\cite{Alwall:2014hca} for 2016 (2017 and 2018) samples. 
 Events arising from $\ttbar$ production~\cite{Frixione:2007nw} are a relatively minor background and are generated at NLO with \POWHEG v2.0~\cite{cite:powheg1,cite:powheg2}.
 The background from vector boson scattering (VBS) production of $\PZ/\gamma^{*}$+$\gamma$ pairs, with the $\PZ$ boson decaying to a pair of leptons, is simulated at leading order using the \MGvATNLO generator. The decay $\PH\to\mu^+\mu^-$ is considered as a resonant background and is generated for the $\Pg\Pg\PH$, VBF,  $\mathrm{V}\PH$, and $\ttbar\PH$ production mechanisms. The $\Pg\Pg\PH$ production cross section is
computed at next-to-next-to-NLO precision in QCD and at NLO in electroweak (EWK)
theory~\cite{Anastasiou:2016cez}. 
The cross sections for Higgs boson production in the VBF~\cite{PhysRevLett.115.082002} and VH~\cite{BREIN2004149} mechanisms are calculated at next-to-NLO in QCD, including NLO EWK corrections, while the $\ttbar\PH$ cross section is computed at NLO in QCD and EWK theory~\cite{PhysRevD.68.034022}. 

A full list of simulated physics processes used in the analysis is provided in Table \ref{tab:sim_samples}, along with cross section and branching fraction information. 
For the \hzg{} and $\PH\to\mu^+\mu^-$ samples, the SM Higgs boson production cross sections and branching fractions
recommended by the LHC Higgs Working
Group~\cite{LHC-YR4} are used for each mass point.
The values for the Higgs boson processes listed in Table \ref{tab:sim_samples} correspond to $m_{\PH}=125\GeV$.

\begin{table}[tb]
	\begin{center}
		\caption{Simulated event samples used in the analysis. For Higgs samples, the listed cross sections and branching fractions correspond to the values for $m_{\PH}=125\GeV$.}
		\begin{tabular}{|l|c|c|}
			\hline
			\textbf{process} & \textbf{cross section (pb)} & \textbf{branching fraction}\\\hline 
			$\mathrm{gg}\PH\to\PZ\gamma\to\lplm\gamma$ & 48.58 & $1.548\times 10^{-3}$ \\ 
			$\mathrm{q\bar{q}}\PH\to\PZ\gamma\to\lplm\gamma$ & 3.782 & $1.548\times 10^{-3}$\\
			$\PZ\PH\to\PZ\gamma\to\lplm\gamma$ & 0.8839 & $1.548\times 10^{-3}$\\ 
			$\PW^+\PH\to\PZ\gamma\to\lplm\gamma$ & 0.84 & $1.548\times 10^{-3}$\\
			$\PW^-\PH\to\PZ\gamma\to\lplm\gamma$ & 0.532 & $1.548\times 10^{-3}$\\ 
			$\ttbar\PH\to\PZ\gamma\to\lplm\gamma$ & 0.5071 & $1.548\times 10^{-3}$\\ 
			$\mathrm{gg}\PH\to\mpmm$ & 48.58 & $2.176\times 10^{-4}$\\ 
			$\mathrm{q\bar{q}}\PH\to\mpmm$ & 3.782 & $2.176\times 10^{-4}$\\
			$\PZ\PH\to\mpmm$ & 0.8839 & $2.176\times 10^{-4}$\\ 
			$\PW^+\PH\to\mpmm$ & 0.84 & $2.176\times 10^{-4}$\\
			$\PW^-\PH\to\mpmm$ & 0.532 & $2.176\times 10^{-4}$\\ 
			$\ttbar\PH\to\mpmm$ & 0.5071 & $2.176\times 10^{-4}$\\ 
			$\PZ/\gamma^*\to\lplm\gamma$ & 117.864 & --\\
			$\PZ/\gamma^*\to\lplm$+jets & 6225.42 & --\\
			$\ttbar$ (inclusive) & 831.76 & --\\
			VBF $\PZ/\gamma^*+\gamma$ & 0.1097 & --\\

			\hline
		\end{tabular}
		\label{tab:sim_samples}
	\end{center}
\end{table}

All simulated events are interfaced
with \PYTHIA v8.226~(v8.230)~\cite{Sjostrand:2014zea} with the
CUETP8M1~\cite{Khachatryan:2015pea} (CP5~\cite{Sirunyan:2019dfx}) underlying event tune for 2016 (2017--2018) for the
fragmentation and hadronization of partons and the internal bremsstrahlung of the leptons. The NLO parton distribution function (PDF) set, NNPDF v3.0~\cite{nnpdf30}~(NNPDF v3.1)~\cite{nnpdf_new}, is used to produce these samples in 2016 (2017--2018). The response of the CMS detector is modeled using the
\GEANTfour  program~\cite{AGOSTINELLI2003250}. 
The simulated events are reweighted to correct for differences between data and simulation in the number of additional $\Pp\Pp$ interactions, trigger efficiencies, selection efficiencies, and efficiencies of isolation requirements for photons, electrons, and muons. These corrections and their associated uncertainties will be described in more detail throughout the remainder of this thesis. The rest of this chapter will discuss general procedures and corrections related to the data and simulated samples.

\section{Z+jets Overlap Removal}
The two dominant backgrounds in this analysis are $\PZ/\gamma^{*}(\rightarrow \ell^+\ell^-)$+$\gamma$, or $\PZ\gamma$ for short, and $\PZ/\gamma^{*}(\rightarrow \ell^+\ell^-)$+jets, or $\PZ$+jets for short. Both processes 
are simulated independently in order to obtain the best model of the kinematics of the $\lplm\gamma$ system. 
The $\PZ\gamma$ sample simulates photons from initial-state and final-state radiation at the matrix element level and includes a better 
computation of the interference between the corresponding Feynman diagrams than the $\PZ$+jets sample.
However, the $\PZ$+jets
simulated sample also includes the contribution from $\PZ\gamma$. Therefore, this contribution must be removed from the 
$\PZ$+jets sample in order to avoid double counting events and to utilize the better kinematic modeling of the $\PZ\gamma$
sample. To handle this, $\PZ$+jets events are removed from the analysis if they contain prompt, final state photons within 
$\DR = \sqrt{\smash[b]{(\Delta\phi)^2 + (\Delta\eta)^2}} = 0.1$ of the photon selected to reconstruct the $\lplm\gamma$ system.
The events with prompt, final state photons are tagged using generator-level truth information.

\section{Photon Internal Conversion}
\label{sec:gconversion}
The PYTHIA 8 showering algorithm automatically 
includes a component in which $\PH\to\PGg\PGg$ photons are 
internally converted via the process $\PH\to\PGg\PGg*\to \mathrm{f\bar{f}}\PGg\PGg$, with $\mathrm{f\bar{f}}$ mostly $\epem$. 
This also affects the $\PH\to \PZ\PGg$ signal and $\PZ\PGg$ background samples. 
Since the normalization of these samples does not include the internal conversion component, the cross section must be adjusted accordingly.
To determine the amount by which the cross section is adjusted, we measure the internal conversion
rate in simulation. 
We define a ratio where the numerator is the number of events with hard process final state photons
and the denominator is total number of events in the simulation. Subtracting this ratio from unity gives the internal conversion rate.
The hard process final state photon flagging is done using generator-level truth information.
In our analysis, the internal conversion rate is about 3.1\% (2016) and 3.0\% (2017 and 2018).  
We, therefore, divide the affected sample cross sections by 0.969 (0.97) for 2016 (2017 and 2018).

\section{Transverse Momentum Reweighting}\label{sec:Zpt}
For the 2016 MC, the underlying event tuning was developed in Run-1 with 7 \TeV data. 
To get a more appropriate underlying description, the GEN group provided a new tuning
for 2017 and 2018 MC production, updating from CUETP8M1 (2016) to CP5 (2017/2018).
However, a discrepency was found in the $\ell\ell\gamma$ \pt spectrum
which had an obvious and significant impact in the SM $Z\gamma$ sample. 
Since $\pt^{ll\gamma}/m_{ll\gamma}$ is an important variable in the kinematic MVA training, 
we correct for this issue by reweighting the $\pt^{ll\gamma}$ from simulation to data in 
2017 and 2018 using a sideband region. 
We define the sideband region as $115GeV<m_{ll\gamma}<120GeV$ and $130GeV<m_{ll\gamma}<135GeV$,
which is chosen close to signal region to avoid differences in kinematics.
We evaluate this weight for the electron and muon channels for 2017 and 2018.
\begin{figure*}[htbp]
	\begin{center}
		\includegraphics[width=0.35\textwidth]{fig/zpt_reweight/zgptrewei_datasb17.pdf}
		\includegraphics[width=0.35\textwidth]{fig/zpt_reweight/zgptrewei_datasb18.pdf}
	\end{center}
	\caption{Simulation before and after the correction to the $\ell\ell\gamma$ \pt.
    The black points show the data. Left:2017, right: 2018.}
\end{figure*}

\section{Pileup Reweighting}\label{sec:pileup}
The simulation includes an accurate distribution of the number of interactions 
taking place in each bunch crossing. Although the Deterministic Annealing primary vertex reconstruction \cite{detanneal} has been shown to
be efficient and well-behaved up to the observed levels of pileup, the final distribution
for the number of reconstructed primary vertices is still sensitive to differences between data and 
MC in the primary vertex reconstruction and underlying event.
Additionally, there is a potentially larger effect where the distribution for the number of
reconstructed vertices can be biased by the offline event selection criteria and even by the trigger.
In order to factorize these effects, instead of reweighting the MC by the number of 
reconstructed Primary Vertices, we reweight the number of pileup interactions in the simulation 
(as stored in the PileupInfo collection in the MC). The pileup distribution for data is
derived by using the per-bunch-crossing-per-luminosity-section instantaneous luminosity from
the LumiDB together with the total pp inelastic cross section. The total pp inelastic cross 
section is taken to be 66 mb, 72.4 mb, and 80 mb for 2016, 2017, and 2018 respectively.
This yields an expected pileup distribution, correctly weighted by the 
per-bunch-crossing-per-luminosity-section integrated luminosity over the entire data-taking period. The inelatistic cross-sections are choosing the  
Figure \ref{fig:puwei} shows the number of primary vertices distribution after 
pile-up reweighting in
the $ee\PGg$ channel. 
\begin{figure}[hbtp]
  \begin{center}
     \includegraphics[width=0.3\textwidth]{fig/pileup/ele_kin_nVtx_valid_Legacy16_HLT.pdf}
     \includegraphics[width=0.3\textwidth]{fig/pileup/ele_kin_nVtx_valid_Rereco17_HLT.pdf}
     \includegraphics[width=0.3\textwidth]{fig/pileup/ele_kin_nVtx_valid_Rereco18_HLT.pdf}
  \end{center}
\caption{Distribution of the number of primary vertices in data and simulation shown for the electron channel after 2 leptons and 1 photon are selected. Left to right: 2016, 2017, 2018. The pileup cross section is taken to be 66 mb, 72.4 mb, and 80 mb for 2016, 2017, and 2018 respectively.}
\label{fig:puwei}
\end{figure}

\section{Level-1 Trigger Prefiring Problem}\label{sec:L1}
In 2016 and 2017, a gradual timing shift of the ECAL was not properly propagated to 
L1 trigger primitives (TP) resulting in a significant fraction of high eta TP 
being mistakenly associated to the previous bunch crossing. 
Since Level 1 rules forbid two consecutive bunch crossings to fire, 
an unpleasant consequence of this (in addition to not finding the TP in the bx 0) 
is that events can self veto if a significant amount of ECAL energy is found in the region 
of $2.<|\eta|<3$. This effect is not described by the simulations \cite{L1_prefire}.\\

The JETMET twiki provides a recipe to compute the probability for an event not to prefire, that can then be applied to the simulations. This is achieved with an EDProducer that runs over all offline photons and jets found in the event and assign them a prefiring probability. 

The final event weight is obtained as the product of the non prefiring probability of all objects (measured using unprefirable events), namely:
\begin{equation}
	\omega = 1 - P(Prefiring) = \prod_{i=photon,jets}(1-\epsilon_i^{pref}(\eta,\pt^{EM}))
\end{equation}

The impact of the L1 prefiring for electrons (muons) is a 1.4 (0.8)\% 
loss in signal yield for 2016 and 2.4 (1.1)\% 
loss in signal yield for 2017.



\chapter{Object and Event Selection}

\section{Triggers}
The topology and basic kinematics of the \hzg process guide the choice of triggers used in this analysis. As this is a three body
decay, the \pT of the photon tends to be less than in other channels like \hgg. Consequently, the CMS photon trigger \pT thresholds 
are too great to make them viable options. Instead, we trigger on the leptons arising from the decay of the Z boson, which tend 
to have larger \pT. The best approach to maximize signal efficiency is to use the double lepton triggers. The double muon trigger 
has \pT thresholds of 17 and 8 GeV, and the double electron trigger has thresholds of 23 and 12 GeV. These are the lowest 
unprescaled double lepton triggers generally available for CMS Run 2 analysis. Events passing both the double muon and 
double electron triggers are treated as double muon events.

The triggers are applied to both data and simulation. Trigger efficiencies and scale factors are measured using the 
simulation samples corresponding to each data-taking year. These measurements use a tag and probe method [CITE]. The tag and probe  method
takes advantage of the high purity of $Z \rightarrow \ell^{+}\ell{-}$ events near the Z mass peak. One lepton functions as the 
tag, and satisfies a set of tight trigger, identification, isolation, \pT requirements. The second lepton, the probe, must pass a looser selection and is used to measure the efficiency in question. 
Using this approach, trigger efficiencies for each leg of a given 
double lepton trigger are measured in both data and simulation. Then a corrective scale factor, defined as the ratio of data efficiency 
to simulation efficiency, is applied to the simulation. Scale factors are measured and applied in bins of \pT and $|\eta|$.

For the double electron trigger efficiency measurements, the tag electron must satisfy the following requirements. It must pass 
the single electron trigger, pass a tight cut-based identification, have \pT $> 30 \, (35)$ in 2016 (2017/2018), and have $|\eta| < 2.5$. 
The probe electron must pass the loose electron MVA identification (with isolation) requirement. The efficiencies for each leg 
of the trigger are measured separately, so in each case, the probe electron must match the trigger leg being measured. The efficiencies 
and scale factors of each double electron trigger leg for 2016, 2017, and 2018 are shown in Figures [FIGS]. 

For the double muon trigger efficiency measurements, the tag muon must satisfy the following requirements. It must pass the single 
muon trigger, pass a tight cut-based identification, have \pT $> 26 \, (29)$ GeV for 2016 (2017/2018) and satisfy $|\eta| < 2.4$. The 
probe muon must pass the $H \rightarrow ZZ$ identification and isolation cuts. The details of the $H \rightarrow ZZ$ muon 
identification will be described later. The efficiencies for each leg of the trigger are measured separately, so in each case, the probe 
muon must match the trigger leg being measured. The efficiencies and scale factors of each double muon trigger leg for 2016, 2017, 
and 2018 are shown in Figures [FIGS]. 

\section{Muon Selection}
A loose cut-based muon identification is used in the analysis. This identification was originally developed by the \hzz 
analysis [REF] and is well-suited for \hzg due to the similarity of the multiple, potentially soft, muons in the final state. 
All muons are first required to pass a set of common cuts, followed by a separate set of cuts for high \pT (greater than 200 GeV) 
and low \pT muons. All muons must satisfy \pT $>$ 5 GeV, $|\eta| < 2.4$, $|d_{xy}| < 0.5$ cm, and $|d_{z}| < 1$ cm, where $d_{xy}$ and 
$d_{z}$ are impact parameters defined with respect to the primary vertex of the interaction using the best muon track. 
Additionally, the three dimensional impact parameter (analogously defined) must have a magnitude less than four times its uncertainty.
Muons must either be reconstructed as global muons or tracker muons. Muons with standalone tracks (tracks only in the muon system) are
rejected. Muons must pass a particle flow-based isolation requirement, where the relative particle flow isolation 
within a $\Delta R = 0.3$ cone is defined as 
\begin{equation}
\label{eqn:pfiso}
	\mathcal{I} \equiv \Big( \sum p_{T}^\text{charged} + \max\big[ 0, \sum p_{T}^\text{neutral} +\sum p_{T}^{\gamma} - p_{T}^\mathrm{PU}(\ell) \big] \Big) / p_{T}^{\ell}.
\end{equation}
Muons must satisfy $\mathcal{I} < 0.35$. 

For muons with \pT $<$ 200 GeV, muons satisfying the common requirements and identified by the particle flow identification
algorithm are selected. For muons with \pT $>$ 200 GeV, muons are selected if they pass the particle flow identification or if they pass
a set of high-$p_{T}$ requirements. These requirements are the following: the muon must be matched to segments in at least two muon 
stations; satisfy $\frac{p_{T}}{\sigma_{p_{T}}} < 0.3$, $|d_{xy}| < 0.2$ cm, $|d_{z}| < 0.5$ cm, have at least one pixel hit, and have 
tracker hits in at least six tracker layers.

In 2016, data-taking was affected by a problem in which the level one trigger sent only one candidate per 60$^{\circ}$ sector instead 
of up to three [REF?]. As a result, when two muons in the same endcap had a low $\Delta \phi$ separation, only one would fire the 
trigger. To account for this, 2016 data events containing identified muons with $\Delta \phi < 70^{\circ}$ in the same endcap region 
are rejected. 

Identification efficiencies and scale factors corresponding to the muon identification above are measured and provided by the 
\hzz analysis working group, and are shown in Figure [FIG]. The momentum of each muon passing the identification requirements is 
corrected using the Rochester method and corrections [REF]. 



\section{Electron Selection}
Electrons are identified using a boosted decision tree multivariate (MVA) discriminator trained on Drell-Yan plus jets simulation 
with prompt electrons matched to generator-level objects as signal and unmatched and non-prompt electrons as background. The features 
used in the training include \pT, supercluster $\eta$, shower shape variables, ratio of hadronic to electromagnetic energy, track and 
pixel hit variables, and isolation variables [REF]. These features are sensitive to bremsstrahlung along the electron trajectory, 
momentum-energy matching between electron trajectory and ECAL cluster, shower shape, and electrons from photon conversions.
Since isolation features are included in the training of the discriminator, there is no need for separate isolation cuts. 
Electrons pass the identification requirement of the \hzg analysis if their discriminator score is higher than a loose working point 
value, corresponding to 98\% signal efficiency. In addition to the MVA cut, electrons must have $|d_{xy}| < 0.5$ cm and $|d_{z}| < 1$ cm
with respect to the primary vertex. Electrons with \pT $\leq$ 7 GeV are rejected. 

Electron identification efficiencies and scale factors are measured using a tag and probe method on dielectron events near the 
Z boson peak. These electrons must pass the single electron trigger with a \pT threshold of 27 (32) GeV for 2016 (2017/2018) data. 
The dielectron mass must be between 60 and 120 GeV. The tag electron must pass a tight cut-based electron identification and have 
\pT $>$ 30 (35) GeV for 2016 (2017) and $|\eta|$ < 2.5. The probe electron must pass the loose MVA identification cut and associated 
impact parameter and \pT cuts described above. Identification efficiencies and scale factors are measured and applied in bins of 
\pT and supercluster $\eta$. The scale factors for each data-taking year are shown in Figures [FIGS]. 

\section{Photon Selection}


\chapter{Event Categorization}

The sensitivity of the \hzg search can be significantly improved by leveraging the differences between signal events
arising from different Higgs boson production modes with different final state topologies and kinematics. 
This is achieved by defining mutually exclusive categories targeting these different types of signal events. In previously 
published CMS searches for \hzg [REFS], cut-based approaches were used to define the categories. Broadly, the presence of at least 
one additional lepton was used to tag VH and ttH production, a dijet system was used to tag VBF production, a boosted jet was 
used to tag gluon-gluon fusion events recoiling off of a jet, and properties of the photon were used to divide the remaining 
events into kinematically distinct untagged categories. The specific definitions of the categories used in the 2016 search 
at 13 TeV are shown in Figure [FIG]. 

The categorization of the present search is inspired by the previous CMS searches, but is significantly more sophisticated and 
better optimized than those searches. The lepton tag category remains, as do the concepts of dijet and untagged categories. 
However, the dijet and untagged categories are determined via the training of two BDTs, with category 
boundaries optimized using the resulting BDT scores. Moreover, the boosted category is studied and found to yield no 
improvement over the categories determined by the BDTs, so the boosted category is dropped in the present analysis. We proceed with 
a description of the BDT procedures for the untagged and dijet cases. Then, with our BDTs in hand, we describe the procedure 
for defining an optimal set of categories to be used in the analysis. We show that this categorization procedure is more 
optimal than what has been done in past CMS searches. 

\section{Kinematic BDT}

The kinematic BDT is used to separate \hzg signal events from background events. Its primary purpose is to define the 
untagged categories, which are distinguished primarily by kinematics and the quality of the final state photon. Its secondary 
use is as an input to the dijet BDT (described below) in order to increase signal to background discrimination in the 
dijet BDT training. The kinematic BDT is trained on \hzg signal events from all Higgs production modes and background events from 
SM $Z\gamma$, Z+jets, and $t\bar{t}$. The training is restricted to use half of the simulated events, with the other half 
reserved for category optimization (described later). All training events are required to pass the basic object and 
event selection described in Chapter 6. Events from 2016, 2017, and 2018 simulation samples in both the muon and electron channels 
are combined for the training. These samples are weighted by their respective cross sections and weighted by the luminosity of
each year. 

\section{Dijet BDT}

\section{Categorization Procedure}

\section{Dropping of Boosted Category}

\section{Comparison to Previous Approach}


\chapter{Statistical Analysis}

The signal search is performed using a simultaneous fit to the $m_{\ell^+\ell^-\gamma}$ distribution in the eight event categories described in Section~\ref{sec:class}.
The expected SM $\PH\to\PZ\gamma$ distributions, scaled by a factor of 10, are also shown.
The fit uses a binned maximum likelihood method in the range $105 < m_{\ell^+\ell^-\gamma} < 170$\GeV.
In each category, a likelihood function is defined using analytic models of signal and background events, along with nuisance parameters for systematic uncertainties.
The combined likelihood function is the product of the likelihood functions in each category.
The parameter of interest in the maximum likelihood fit is the signal strength $\mu$, defined as the product of the cross section and the branching fraction [$\sigma(\Pp\Pp\to\PH)\mathcal{B}(\PH\to\PZ\gamma)$], relative to the SM expectation.

The signal model is defined as the sum of Crystal Ball~\cite{CB-Oreglia} and Gaussian functions.
The signal shape parameters are determined by fitting this model to simulated signal events in each category.
To account for differences in mass resolution, these fits are performed separately for the event samples used to model each data-taking year, as well as for muon and electron channel events.
This results in six signal models that are summed to give the total signal expectation in a given category.
Table~\ref{tab:yield} gives these mass resolutions for $\PH\to\PZ\gamma$, summed over the three years, as obtained from simulation. The mass resolutions range from 1.4--2.3\GeV, depending on the category.
Separate sets of parameter values are found by fitting simulated events with $m_\PH$ of $120$, $125$, and $130$\GeV.
Using linear interpolation, parameter values are also determined at 1\GeV intervals in $m_\PH$ from $120$--$130$\GeV, as well as at \mH\GeV. 
In the fit to data, the mean and resolution parameters are allowed to vary subject to constraints from several systematic uncertainties, described in Section~\ref{sec:systematics}, while the remaining parameters are held fixed.

The background model in each category is obtained from the data using the discrete profiling method~\cite{Dauncey:2014xga}.
This technique accounts for the systematic uncertainty associated with choosing an analytic functional form to fit the background.
The background function is chosen from a set of candidate functions via a discrete nuisance parameter in the fit.
These functions are derived from the data in each category, with muon and electron events from all data-taking years combined.
The $m_{\ell^+\ell^-\gamma}$ spectrum consists of a turn-on peak around 110--115\GeV and a falling spectrum in the high-mass tail, where the turn-on peak is driven by the photon $\pt$ selection.
These features are modeled by the convolution of a Gaussian function with a step function multiplied by one of several falling spectrum functions.
The complete function has the general form:
\begin{equation}
    \mathcal{F}(m_{\ell^+\ell^-\gamma}; \mu_{\mathrm{G}}, \sigma_{\mathrm{G}}, s, \vec{\alpha}) = \int_{105}^{170}\mathcal{N}(m_{\ell^+\ell^-\gamma}-t;\mu_{\mathrm{G}},\sigma_{\mathrm{G}})\Theta(t; s)f(t; \vec{\alpha})dt,
\end{equation}
where $t$ is the integration variable for the convolution, $\mathcal{N}(m_{\ell^+\ell^-\gamma}-t;\mu_{\mathrm{G}},\sigma_{\mathrm{G}})$ is the Gaussian function with mean $\mu_{\mathrm{G}}$ and standard deviation $\sigma_{\mathrm{G}}$, $\Theta(t; s)$ is the Heaviside step function with step location $s$, and $f(t; \vec{\alpha})$ is the falling spectrum function with shape parameters $\vec{\alpha}$.
The falling spectrum function families considered include exponential functions, power law functions, Laurent series, and Bernstein polynomials.
Functions from each family are selected based on a chi-squared goodness-of-fit criterion ($\text{\textit{p}-value} > 0.01$) as well as an $\mathcal{F}$-test~\cite{Fisher:1922saa}, which determines the highest order function to be used.
A penalty term is added to the final likelihood to take into account the number of parameters in each function, ensuring that higher-order functions will not be preferred a priori.
The set of profiled background functions in each category is checked to ensure that any bias introduced into the fit results is small and that the associated CL intervals have the appropriate frequentist coverage. For each function, pseudo-data sets are generated under a fixed signal strength hypothesis. A signal-plus-background fit is performed on each pseudo-data set, with the choice of background function profiled, and the distribution of best fit signal strengths is used to assess the bias and frequentist coverage.

The best fit value of the signal strength, $\hat{\mu}$, is determined by maximizing the likelihood, accounting for all nuisance parameters.
The uncertainty in $\hat{\mu}$ and the observed significance are derived from the profile likelihood test statistic~\cite{cite:l3},
\begin{equation}
	q(\mu) = -2\mathrm{ln}\Bigg(\frac{\mathcal{L}(\mu, \hat{\vec{\theta_{\mu}}})}{\mathcal{L}(\hat{\mu},\hat{\vec{\theta}})}\Bigg),
\end{equation}
where $\vec{\theta}$ is the set of nuisance parameters, $\hat{\mu}$ and $\hat{\vec{\theta}}$ are unconditional best fit values, and $\hat{\vec{\theta_{\mu}}}$ is the set of conditional best fit values of the nuisance parameters for a given value of $\mu$.
An upper limit on $\mu$ is determined using the profile likelihood statistic with the \CLs criterion.
The asymptotic approximation for the sampling distribution of $q(\mu)$ is assumed in the derivation of these results~\cite{cite:l1,cite:l2,cite:l3,Cowan:2010js}.
The expected significance under the SM hypothesis and the expected upper limits under the background-only hypothesis are also reported.
These are obtained by fitting to the corresponding Asimov data sets~\cite{Cowan:2010js}.

In addition, a combined maximum likelihood fit with the CMS measurement~\cite{CMS:2021kom} of $\PH\to\PGg\PGg$ using the same data sample is performed to determine the ratio $\mathcal{B}(\PH\to\PZ\gamma)/\mathcal{B}(\PH\to\gamma\gamma)$.
The $\PH\to\PGg\PGg$ analysis obtained a signal strength for $\sigma(\Pp\Pp\to\PH)\mathcal{B}(\PH\to\PGg\PGg)$ of $1.12\pm0.09$.
In this combined fit, the branching fraction $\mathcal{B}(\PH\to\gamma\gamma)$ is an additional free parameter.
The uncertainty in the measured ratio of the two branching fractions is dominated by the statistical uncertainties in the branching fractions.
Common sources of theoretical and experimental uncertainty in the two measurements, described in the next section, are treated as correlated in the fit.
The combination is performed at $m_\PH = \mH$\GeV, and the discrete profiling method is used for the background modeling in both cases.


\chapter{Systematic Uncertainties}\label{sec:uncertainties}


\chapter{Results and Interpretation}\label{sec:results}

Figures~\ref{fig:3}~and~\ref{fig:4} show the $m_{\ell^+\ell^-\gamma}$ distributions of the data events in each category.
The expected SM $\PH\to\PZ\gamma$ distributions, scaled by a factor of 10, are also shown.
Figure~\ref{fig:SignalBackground} shows the signal-plus-background fit to the data and the corresponding distribution after background subtraction for the sum of all categories. 
Each category is weighted by the factor $S/(S+B)$, where $S$ is the expected signal yield  
and $B$ is the background yield in the narrowest mass interval containing 95\% of the signal distribution.

\begin{figure}
  \centering
  \includegraphics[width=0.45\textwidth]{fig/results/Figure_008.pdf}
  \includegraphics[width=0.45\textwidth]{fig/results/Figure_004-a.pdf}\\
  \includegraphics[width=0.45\textwidth]{fig/results/Figure_004-b.pdf}
  \includegraphics[width=0.45\textwidth]{fig/results/Figure_004-c.pdf}\\
   \caption{Fits to the $m_{\ell^+\ell^-\gamma}$ data distribution
    in the lepton-tagged (upper left), dijet 1 (upper right), dijet 2 (lower left), and
  dijet 3 (lower right) categories.
  In the upper panel, the red solid line shows the result of a signal-plus-background fit to the given category.
  The red dashed line shows the background component of the fit.
  The green and yellow bands represent the $68$ and $95$\% \CL\ uncertainties in the fit.
  Also plotted is the expected SM signal, scaled by a factor of 10.
  In the lower panel, the data minus the background component of the fit is shown. \label{fig:3}}
  \end{figure}

  \begin{figure}
  \centering
  \includegraphics[width=0.45\textwidth]{fig/results/Figure_006-a.pdf}
  \includegraphics[width=0.45\textwidth]{fig/results/Figure_006-b.pdf}\\
  \includegraphics[width=0.45\textwidth]{fig/results/Figure_006-c.pdf}
  \includegraphics[width=0.45\textwidth]{fig/results/Figure_006-d.pdf}
   \caption{Fits to the $m_{\ell^+\ell^-\gamma}$ data distribution
    in the untagged 1 (upper left), untagged 2 (upper right), untagged 3 (lower left), and
  untagged 4 (lower right) categories.
  In the upper panel, the red solid line shows the result of a signal-plus-background fit to the given category.
  The red dashed line shows the background component of the fit.
  The green and yellow bands represent the $68$ and $95$\% \CL\ uncertainties in the fit.
  Also plotted is the expected SM signal, scaled by a factor of 10.
  In the lower panel, the data minus the background component of the fit is shown. \label{fig:4}}
  \end{figure}

\begin{figure}
\centering
\includegraphics[width=0.5\textwidth]{fig/results/Figure_012.pdf}
 \caption{Sum over all categories of the data points and signal-plus-background model after the simultaneous fit to each $m_{\ell^+\ell^-\gamma}$ distribution. 
 The contribution from each category is weighted by $S/(S+B)$, as defined in the text. 
 In the upper panel, the red solid line shows the signal-plus-background fit. The red dashed line shows the background component of the fit. The green and yellow bands represent the $68$ and $95$\% CL uncertainties in the fit. Also plotted is the expected SM signal weighted by $S/(S+B)$ and scaled by a factor of 10. In the lower panel, the data minus the background component of the fit is shown.
   }
\label{fig:SignalBackground}
\end{figure}

The best fit value of the signal strength is $\signalstrengthExpanded$ at $m_\PH=\mH$\GeV.
The measured value of $\sigma(\Pp\Pp\to\PH)\mathcal{B}(\PH\to\PZ\gamma)$ is $\brExpanded$\,pb. This measurement is consistent with the SM prediction of $0.09 \pm 0.01$\,pb at the \compatibility\, standard deviation level.
Figure~\ref{fig:lim-combo125} shows the signal strengths obtained for each category separately, corresponding to the fit results shown in Figs.~\ref{fig:3}~and~\ref{fig:4}, as well as from simultaneous fits to the dijet categories, the untagged categories, and all categories combined. 
Among the eight categories, dijet 1 is the most sensitive.
A category compatibility $\textit{p}$-value, under the hypothesis of a common signal strength in all categories, is calculated from the likelihood ratio between the 
nominal combined fit, in which all categories have the same signal strength parameter, 
and a separate fit, in which each category has its own signal strength parameter. 
This $\textit{p}$-value is found to be \channelcompatp, corresponding to \channelcompatsigma\, standard deviations, and is driven by the dijet 3 category, which has a signal strength of $\hat{\mu}=\signalstrengthdijetthree$. 
The observed (expected) local significance is \obssig\,(\expsig) standard deviations. 
Upper limits on $\mu$ are calculated at 1\GeV intervals 
in the mass range of $120 < m_{\ell^+\ell^-\gamma} < 130\GeV$ and at $m_\PH=\mH\GeV$, as shown in Fig.~\ref{fig:lim}.
The observed (expected) limit at 95\% CL relative to the SM expectation for $m_\PH=\mH\GeV$ is $\obslimit$ ($\explimit$). 

\begin{figure}
  \centering
  %  \includegraphics[width=0.75\textwidth]{Figure_010.pdf}
   \includegraphics[width=0.7\textwidth]{fig/results/Figure_011.pdf}
    \caption{
Observed signal strength ($\mu$) for a SM Higgs boson with $m_\PH=\mH\GeV$. 
The labels ``untagged combined," ``dijet combined," and ``combined" represent the results obtained from simultaneous fits of the untagged categories, dijet categories, and full set of categories, respectively. 
%Fits perform simultaneously for the combined categories, where the untagged, dijet and all categories are considered. 
The black solid line shows $\mu=1$, and the red dashed line shows the best fit value $\hat{\mu}=\,$\signalstrength ~of all categories combined.
    \label{fig:lim-combo125}}
\end{figure}

\begin{figure}
  \centering
  \includegraphics[width=0.7\textwidth]{fig/results/Figure_009.pdf}
    \caption{
	    Upper limit ($95$\%~\CL) on the signal strength ($\mu$) relative to the SM prediction, as a function of the assumed value of the Higgs boson mass used in the fit.
    \label{fig:lim}}
\end{figure}

\section{Ratio Measurement}
The ratio $\mathcal{B}(H\rightarrow\PZ\gamma)/\mathcal{B}(H\rightarrow\gamma\gamma)$ is theoretically interesting, as it is 
potentially sensitive to BSM physics, as described in Chapter~\ref{sec:intro}.
By combining the \hzg{} analysis with the existing CMS $\PH\to\gamma\gamma$ analysis of the same data set~\cite{CMS:2021kom}, 
we can obtain a measurement of this ratio and compare it to the SM prediction of $0.69 \pm 0.04$. 

The profile likelihood scans for the parameters of interest in the ratio fit are shown in Figure~\ref{fig:scan_br}. The best fit values of the 
parameters of interest are $\mu_{\gamma\gamma} = 1.121^{+0.095}_{-0.090}$ and 
$\mu_{Z\gamma}/\mu_{\gamma\gamma} = 2.225^{+0.926}_{-0.825}$. 
The value of $\mu_{\gamma\gamma}$ agrees well with the standalone $\PH\to\gamma\gamma$ fit. 
Multiplying through, the value of $\mu_{Z\gamma}$ is 2.49, which 
agrees well with the result of the standalone \hzg{} fit.
The measured value of $\mathcal{B}(\PH\to\PZ\gamma)/\mathcal{B}(\PH\to\gamma\gamma)$ is \brRatio. 
This measurement is consistent with the SM prediction for the ratio at the \brRatioCompat\, standard deviation level. 

\begin{figure}
   \begin{center}
   \includegraphics[width=0.45\textwidth]{fig/results/ratio/scan_mu_BR_gamgam.pdf}
   \includegraphics[width=0.45\textwidth]{fig/results/ratio/scan_mu_BR_Zgam_r_BR_gamgam.pdf}\\
   \caption{Profile likelihood scans of $\mu_{\gamma\gamma}$ and $\mu_{\PZ\gamma}/\mu_{\gamma\gamma}$ for $m_H=125.38\GeV$.}
   \label{fig:scan_br}
   \end{center}    
\end{figure}


\chapter{Conclusion}\label{sec:conclusion}
A search is performed for a standard model (SM) Higgs boson decaying into a lepton pair ($\Pe^{+}\Pe^{-}$ or $\mu^{+}\mu^{-}$) and a photon with $m_{\ell^+\ell^-}>50$ \GeV. 
The analysis is performed using a sample of CMS proton-proton ($\Pp\Pp$) collision data at $\sqrt{s}=$13\TeV, corresponding to an integrated
luminosity of \LumiT\fbinv. 
The main contribution to this final state is 
from Higgs boson decays to a $\PZ$ boson and a photon ($\PH\to\PZ\gamma\to\ell^+\ell^-\gamma$). 
The contributions from final-state radiation of $\PH\to\mu^{+}\mu^{-}$ or $\PH\to\tau^{+}\tau^{-}$  
are at the $6$ and $<1$\% levels, respectively, and are not included in the measured $\PH\to\PZ\gamma$ signal.
The best fit value of the signal strength $\hat{\mu}$ for $m_\PH=\mH\GeV$ is $\hat{\mu}=\signalstrengthExpanded=\signalstrength$.  
This measurement corresponds to $\sigma(\Pp\Pp\to\PH)\mathcal{B}(\PH\to\PZ\gamma)=\,$\br\,pb. 
The measured value is \compatibility\, standard deviations higher than the SM prediction.
The observed (expected) local significance is \obssig\,(\expsig) standard deviations, where the expected significance is determined for the SM hypothesis.
The observed (expected) upper limit at $95$\% confidence level on $\mu$ is \obslimit\,(\explimit). 
In addition, a combined fit with the $\PH\to\PGg\PGg$ analysis of the same data set~\cite{CMS:2021kom} is performed to measure 
the ratio $\mathcal{B}(\PH\to\cPZ\gamma)/\mathcal{B}(\PH\to\gamma\gamma)=\brRatio$, which is consistent
with the ratio of $0.69 \pm 0.04$ predicted by SM at the \brRatioCompat\, standard deviation level for a Higgs boson mass of $\mH$\GeV.



% Uncomment the 'singlespace' environment and '\bibsep' command
% if needed - some bibliographic styles overide the definition
% of 'thebibliography' in nuthesis.cls
%
%\begin{singlespace}
%\bibsep 12pt
\clearpage\phantomsection % needed for hyperlinks to work correctly
%\begin{thebibliography}{xxx}
%
%
%\bibitem{label1} A bibliographic item.  A bibliographic item.  A
%bibliographic item.  A bibliographic item.
%\bibitem{label2} Another bibliographic item.  
%\bibitem{label3} Yet another bibliographic item.  
% The usages of \bibitem and \cite{..} are 
% explained in Section 4.3 (page 73) of % LaTeX manual.
% Or you may use BibTeX.
%\end{thebibliography}
%\end{singlespace}
% (The following suggested by Francisco Iacobelli - 5/11/2010)
% In case, you want to use BibTeX, you should replace (or comment)
% the bibliography environment.
% Instead uncomment the following lines and replace <bib file>
% with your .bib file:
 \begin{singlespace}
%\bibsep 12pt
 \bibliographystyle{thesis_bibstyle} %or another suitable style.
 \bibliography{refs}
 \end{singlespace}

\appendix		% Appendix begins here (optional).

\chapter{Shower Shape Corrections}\label{sec:shower_shape}
For each shower shape variable, the three plots at the top (bottom) correspond to the barrel (endcaps). From left to right, the plots represent 2016, 2017, and 2018. All distributions are from $\PZ\to\epem$ events.
\begin{figure}[htb]
	\centering
	\includegraphics[width=0.25\textwidth]{fig/ss_corr/phoPFPhoIso_16_EB_Z.pdf}
	\includegraphics[width=0.25\textwidth]{fig/ss_corr/ph_phoIso_17_EB_Z.pdf}
	\includegraphics[width=0.25\textwidth]{fig/ss_corr/phoPFPhoIso_18_EB_Z.pdf}\\
	\includegraphics[width=0.25\textwidth]{fig/ss_corr/phoPFPhoIso_16_EE_Z.pdf}
	\includegraphics[width=0.25\textwidth]{fig/ss_corr/ph_phoIso_17_EE_Z.pdf}
	\includegraphics[width=0.25\textwidth]{fig/ss_corr/phoPFPhoIso_18_EE_Z.pdf}\\	
	\label{fig:phoiso_Z}
	\caption{Shower shape correction for the photon isolation.}
\end{figure}
\begin{figure}[htb]
	\centering
	\includegraphics[width=0.25\textwidth]{fig/ss_corr/phoPFChIso_16_EB_Z.pdf}
	\includegraphics[width=0.25\textwidth]{fig/ss_corr/ph_chIso_17_EB_Z.pdf}
	\includegraphics[width=0.25\textwidth]{fig/ss_corr/phoPFChIso_18_EB_Z.pdf}\\
	\includegraphics[width=0.25\textwidth]{fig/ss_corr/phoPFChIso_16_EE_Z.pdf}
	\includegraphics[width=0.25\textwidth]{fig/ss_corr/ph_chIso_17_EE_Z.pdf}
	\includegraphics[width=0.25\textwidth]{fig/ss_corr/phoPFChIso_18_EE_Z.pdf}\\	
	\label{fig:chiso_Z}
	\caption{Shower shape correction for the charged hadron isolation.}
\end{figure}
\begin{figure}[htb]
	\centering
	\includegraphics[width=0.25\textwidth]{fig/ss_corr/phoPFChWorstIso_16_EB_Z.pdf}
	\includegraphics[width=0.25\textwidth]{fig/ss_corr/ph_chWorIso_17_EB_Z.pdf}
	\includegraphics[width=0.25\textwidth]{fig/ss_corr/phoPFChWorstIso_18_EB_Z.pdf}\\
	\includegraphics[width=0.25\textwidth]{fig/ss_corr/phoPFChWorstIso_16_EE_Z.pdf}
	\includegraphics[width=0.25\textwidth]{fig/ss_corr/ph_chWorIso_17_EE_Z.pdf}
	\includegraphics[width=0.25\textwidth]{fig/ss_corr/phoPFChWorstIso_18_EE_Z.pdf}\\	
	\label{fig:chwsiso_Z}
	\caption{Shower shape correction for the charged hadron isolation with worst vertex.}
\end{figure}
\begin{figure}[htb]
	\centering
	\includegraphics[width=0.25\textwidth]{fig/ss_corr/phoEtaWidth_16_EB_Z.pdf}
	\includegraphics[width=0.25\textwidth]{fig/ss_corr/ph_sc_etaWidth_17_EB_Z.pdf}
	\includegraphics[width=0.25\textwidth]{fig/ss_corr/phoEtaWidth_18_EB_Z.pdf}\\
	\includegraphics[width=0.25\textwidth]{fig/ss_corr/phoEtaWidth_16_EE_Z.pdf}
	\includegraphics[width=0.25\textwidth]{fig/ss_corr/ph_sc_etaWidth_17_EE_Z.pdf}
	\includegraphics[width=0.25\textwidth]{fig/ss_corr/phoEtaWidth_18_EE_Z.pdf}\\	
	\label{fig:etawidth_Z}
	\caption{Shower shape correction for the $\eta$ width.}
\end{figure}

\begin{figure}[htb]
	\centering
	\includegraphics[width=0.25\textwidth]{fig/ss_corr/phoPhiWidth_16_EB_Z.pdf}
	\includegraphics[width=0.25\textwidth]{fig/ss_corr/ph_sc_phiWidth_17_EB_Z.pdf}
	\includegraphics[width=0.25\textwidth]{fig/ss_corr/phoPhiWidth_18_EB_Z.pdf}\\
	\includegraphics[width=0.25\textwidth]{fig/ss_corr/phoPhiWidth_16_EE_Z.pdf}
	\includegraphics[width=0.25\textwidth]{fig/ss_corr/ph_sc_phiWidth_17_EE_Z.pdf}
	\includegraphics[width=0.25\textwidth]{fig/ss_corr/phoPhiWidth_18_EE_Z.pdf}\\	
	\label{fig:phiwidth_Z}
	\caption{Shower shape correction for the $\phi$ width.}
\end{figure}
\begin{figure}[htb]
	\centering
	\includegraphics[width=0.25\textwidth]{fig/ss_corr/phoR9_16_EB_Z.pdf}
	\includegraphics[width=0.25\textwidth]{fig/ss_corr/ph_full5x5x_r9_17_EB_Z.pdf}
	\includegraphics[width=0.25\textwidth]{fig/ss_corr/phoR9_18_EB_Z.pdf}\\
	\includegraphics[width=0.25\textwidth]{fig/ss_corr/phoR9_16_EE_Z.pdf}
	\includegraphics[width=0.25\textwidth]{fig/ss_corr/ph_full5x5x_r9_17_EE_Z.pdf}
	\includegraphics[width=0.25\textwidth]{fig/ss_corr/phoR9_18_EE_Z.pdf}\\	
	\label{fig:r9_Z}
	\caption{Shower shape correction for the variable $\mathrm{R}_{9}$.}
\end{figure}
\begin{figure}[htb]
	\centering
	\includegraphics[width=0.25\textwidth]{fig/ss_corr/s4_16_EB_Z.pdf}
	\includegraphics[width=0.25\textwidth]{fig/ss_corr/ph_s4_17_EB_Z.pdf}
	\includegraphics[width=0.25\textwidth]{fig/ss_corr/s4_18_EB_Z.pdf}\\
	\includegraphics[width=0.25\textwidth]{fig/ss_corr/s4_16_EE_Z.pdf}
	\includegraphics[width=0.25\textwidth]{fig/ss_corr/ph_s4_17_EE_Z.pdf}
	\includegraphics[width=0.25\textwidth]{fig/ss_corr/s4_18_EE_Z.pdf}\\	
	\label{fig:s4_Z}
	\caption{Shower shape correction for the variable $\mathrm{S}_4$.}
\end{figure}
\begin{figure}[htb]
	\centering
	\includegraphics[width=0.25\textwidth]{fig/ss_corr/phosieie_16_EB_Z.pdf}
	\includegraphics[width=0.25\textwidth]{fig/ss_corr/ph_sieie_17_EB_Z.pdf}
	\includegraphics[width=0.25\textwidth]{fig/ss_corr/phosieie_18_EB_Z.pdf}\\
	\includegraphics[width=0.25\textwidth]{fig/ss_corr/phosieie_16_EE_Z.pdf}
	\includegraphics[width=0.25\textwidth]{fig/ss_corr/ph_sieie_17_EE_Z.pdf}
	\includegraphics[width=0.25\textwidth]{fig/ss_corr/phosieie_18_EE_Z.pdf}\\	
	\label{fig:sieie_Z}
	\caption{Shower shape correction for the variable $\sigma_{i\eta i\eta}$.}
\end{figure}
\begin{figure}[htb]
	\centering
	\includegraphics[width=0.25\textwidth]{fig/ss_corr/phosieip_16_EB_Z.pdf}
	\includegraphics[width=0.25\textwidth]{fig/ss_corr/ph_sieip_17_EB_Z.pdf}
	\includegraphics[width=0.25\textwidth]{fig/ss_corr/phosieip_18_EB_Z.pdf}\\
	\includegraphics[width=0.25\textwidth]{fig/ss_corr/phosieip_16_EE_Z.pdf}
	\includegraphics[width=0.25\textwidth]{fig/ss_corr/ph_sieip_17_EE_Z.pdf}
	\includegraphics[width=0.25\textwidth]{fig/ss_corr/phosieip_18_EE_Z.pdf}\\	
	\label{fig:sieip_Z}
	\caption{Shower shape correction for the variable $\sigma_{i\eta i\phi}$.}
\end{figure}

\chapter{Signal and Resonant Background Fits}\label{appendix_resonant_fits}

%\section{2017 Signal Fits}
%
%\begin{figure}
%	\begin{center}
%		\includegraphics[width=0.40\textwidth]{fig/signal_fit/2017/sigfit_ele_ggF_1_125.png}
%		\includegraphics[width=0.40\textwidth]{fig/signal_fit/2017/sigfit_ele_ggF_2_125.png}\\
%		\includegraphics[width=0.40\textwidth]{fig/signal_fit/2017/sigfit_ele_ggF_3_125.png}
%		\includegraphics[width=0.40\textwidth]{fig/signal_fit/2017/sigfit_ele_ggF_4_125.png}\\
%		\includegraphics[width=0.40\textwidth]{fig/signal_fit/2017/sigfit_ele_VBF_501_125.png}
%		\includegraphics[width=0.40\textwidth]{fig/signal_fit/2017/sigfit_ele_VBF_502_125.png}\\
%		\includegraphics[width=0.40\textwidth]{fig/signal_fit/2017/sigfit_ele_VBF_503_125.png}\\
%		\caption{Fits to simulated $m_{\ell^+\ell^-\gamma}$ signal distributions in the electron channel for
%            		 $m_\PH=125\GeV$ for the 2017 data-taking period.
%			 The blue line shows the total fit function, the green line shows the Crystal Ball function component, and the red line shows the Gaussian function component.
%			 The top four plots correspond to the untagged categories, and the bottom three plots correspond to the dijet categories.}
%		\label{fig:elesigfit}
%	\end{center}
%\end{figure}
%
%\begin{figure}
%	\begin{center}
%		\includegraphics[width=0.40\textwidth]{fig/signal_fit/2017/sigfit_mu_ggF_1_125.png}
%		\includegraphics[width=0.40\textwidth]{fig/signal_fit/2017/sigfit_mu_ggF_2_125.png}\\
%		\includegraphics[width=0.40\textwidth]{fig/signal_fit/2017/sigfit_mu_ggF_3_125.png}
%		\includegraphics[width=0.40\textwidth]{fig/signal_fit/2017/sigfit_mu_ggF_4_125.png}\\
%		\includegraphics[width=0.40\textwidth]{fig/signal_fit/2017/sigfit_mu_VBF_501_125.png}
%		\includegraphics[width=0.40\textwidth]{fig/signal_fit/2017/sigfit_mu_VBF_502_125.png}\\
%		\includegraphics[width=0.40\textwidth]{fig/signal_fit/2017/sigfit_mu_VBF_503_125.png}\\
%		\caption{Fits to simulated $m_{\ell^+\ell^-\gamma}$ signal distributions in the muon channel for
%            		 $m_\PH=125\GeV$ for the 2017 data-taking period.
%			 The blue line shows the total fit function, the green line shows the Crystal Ball function component, and the red line shows the Gaussian function component.
%			 The top four plots correspond to the untagged categories, and the bottom three plots correspond to the dijet categories.}
%		\label{fig:elesigfit}
%	\end{center}
%\end{figure}
%
%\begin{figure}
%	\begin{center}
%	  \includegraphics[width=0.40\textwidth]{fig/signal_fit/2017/sigfit_ele_mu_ZH_6789_125.png}
%	  \includegraphics[width=0.40\textwidth]{fig/signal_fit/2017/sigfit_ele_mu_WH_6789_125.png}
%		\caption{Fits to simulated $m_{\ell^+\ell^-\gamma}$ signal distributions in the electron and muon channels combined in the lepton-tagged category for
%            		 $m_\PH=125\GeV$ for the 2017 data-taking period.
%        		 The left plot shows the fit to simulated ZH production events, and the right plot shows the fit to simulated WH production events. 
%			 The blue line shows the total fit function, the green line shows the Crystal Ball function component, and the red line shows the Gaussian function component.}
%		\label{fig:elemusigfit}
%	\end{center}
%\end{figure}
%
%\section{2018 Signal Fits}
%
%\begin{figure}
%	\begin{center}
%		\includegraphics[width=0.40\textwidth]{fig/signal_fit/2018/sigfit_ele_ggF_1_125.png}
%		\includegraphics[width=0.40\textwidth]{fig/signal_fit/2018/sigfit_ele_ggF_2_125.png}\\
%		\includegraphics[width=0.40\textwidth]{fig/signal_fit/2018/sigfit_ele_ggF_3_125.png}
%		\includegraphics[width=0.40\textwidth]{fig/signal_fit/2018/sigfit_ele_ggF_4_125.png}\\
%		\includegraphics[width=0.40\textwidth]{fig/signal_fit/2018/sigfit_ele_VBF_501_125.png}
%		\includegraphics[width=0.40\textwidth]{fig/signal_fit/2018/sigfit_ele_VBF_502_125.png}\\
%		\includegraphics[width=0.40\textwidth]{fig/signal_fit/2018/sigfit_ele_VBF_503_125.png}\\
%		\caption{Fits to simulated $m_{\ell^+\ell^-\gamma}$ signal distributions in the electron channel for
%            		 $m_\PH=125\GeV$ for the 2018 data-taking period.
%			 The blue line shows the total fit function, the green line shows the Crystal Ball function component, and the red line shows the Gaussian function component.
%			 The top four plots correspond to the untagged categories, and the bottom three plots correspond to the dijet categories.}
%		\label{fig:elesigfit}
%	\end{center}
%\end{figure}
%
%\begin{figure}
%	\begin{center}
%		\includegraphics[width=0.40\textwidth]{fig/signal_fit/2018/sigfit_mu_ggF_1_125.png}
%		\includegraphics[width=0.40\textwidth]{fig/signal_fit/2018/sigfit_mu_ggF_2_125.png}\\
%		\includegraphics[width=0.40\textwidth]{fig/signal_fit/2018/sigfit_mu_ggF_3_125.png}
%		\includegraphics[width=0.40\textwidth]{fig/signal_fit/2018/sigfit_mu_ggF_4_125.png}\\
%		\includegraphics[width=0.40\textwidth]{fig/signal_fit/2018/sigfit_mu_VBF_501_125.png}
%		\includegraphics[width=0.40\textwidth]{fig/signal_fit/2018/sigfit_mu_VBF_502_125.png}\\
%		\includegraphics[width=0.40\textwidth]{fig/signal_fit/2018/sigfit_mu_VBF_503_125.png}\\
%		\caption{Fits to simulated $m_{\ell^+\ell^-\gamma}$ signal distributions in the muon channel for
%            		 $m_\PH=125\GeV$ for the 2018 data-taking period.
%			 The blue line shows the total fit function, the green line shows the Crystal Ball function component, and the red line shows the Gaussian function component.
%			 The top four plots correspond to the untagged categories, and the bottom three plots correspond to the dijet categories.}
%		\label{fig:elesigfit}
%	\end{center}
%\end{figure}
%
%\begin{figure}
%	\begin{center}
%	  \includegraphics[width=0.40\textwidth]{fig/signal_fit/2018/sigfit_ele_mu_ZH_6789_125.png}
%	  \includegraphics[width=0.40\textwidth]{fig/signal_fit/2018/sigfit_ele_mu_WH_6789_125.png}
%		\caption{Fits to simulated $m_{\ell^+\ell^-\gamma}$ signal distributions in the electron and muon channels combined in the lepton-tagged category for
%            		 $m_\PH=125\GeV$ for the 2018 data-taking period.
%        		 The left plot shows the fit to simulated ZH production events, and the right plot shows the fit to simulated WH production events. 
%			 The blue line shows the total fit function, the green line shows the Crystal Ball function component, and the red line shows the Gaussian function component.}
%		\label{fig:elemusigfit}
%	\end{center}
%\end{figure}
%
%\section{2017 Resonant Background Fits}
%
%\begin{figure}
%	\begin{center}
%		\includegraphics[width=0.40\textwidth]{fig/hmumu/2017/bkgfit_mu_ggF_1_125.png}
%		\includegraphics[width=0.40\textwidth]{fig/hmumu/2017/bkgfit_mu_ggF_2_125.png}\\
%		\includegraphics[width=0.40\textwidth]{fig/hmumu/2017/bkgfit_mu_ggF_3_125.png}
%		\includegraphics[width=0.40\textwidth]{fig/hmumu/2017/bkgfit_mu_ggF_4_125.png}\\
%		\includegraphics[width=0.40\textwidth]{fig/hmumu/2017/bkgfit_mu_VBF_501_125.png}
%		\includegraphics[width=0.40\textwidth]{fig/hmumu/2017/bkgfit_mu_VBF_502_125.png}\\
%		\includegraphics[width=0.40\textwidth]{fig/hmumu/2017/bkgfit_mu_ggF_503_125.png}
%		\caption{Fits to simulated $m_{\mu^+\mu^-\gamma}$ resonant background distributions from $\PH\to\Pgmp\Pgmm$ for
%			 $m_\PH=125\GeV$ for the 2017 data-taking period.
%			 The blue line shows the total fit function, the green line shows the Crystal Ball function component, and the red line shows the Gaussian function component.
%			 The top four plots correspond to the untagged categories, and the bottom three plots correspond to the dijet categories.}
%		\label{fig:mubkgfit}
%	\end{center}
%\end{figure}
%
%\begin{figure}
%	\begin{center}
%		\includegraphics[width=0.40\textwidth]{fig/hmumu/2017/bkgfit_ele_mu_ZH_6789_125.png}
%		\includegraphics[width=0.40\textwidth]{fig/hmumu/2017/bkgfit_ele_mu_WH_6789_125.png}
%		\caption{Fits to simulated $m_{\ell^+\ell^-\gamma}$ resonant background distributions from $\PH\to\Pgmp\Pgmm$ in the electron and muon channels combined in the lepton-tagged category for
%            		 $m_\PH=125\GeV$ for the 2017 data-taking period.
%        		 The left plot shows the fit to simulated ZH production events, and the right plot shows the fit to simulated WH production events. 
%			 The blue line shows the total fit function, the green line shows the Crystal Ball function component, and the red line shows the Gaussian function component.}
%		\label{fig:elemubkgfit}
%	\end{center}
%\end{figure}
%
%\section{2018 Resonant Background Fits}
%
%\begin{figure}
%	\begin{center}
%		\includegraphics[width=0.40\textwidth]{fig/hmumu/2018/bkgfit_mu_ggF_1_125.png}
%		\includegraphics[width=0.40\textwidth]{fig/hmumu/2018/bkgfit_mu_ggF_2_125.png}\\
%		\includegraphics[width=0.40\textwidth]{fig/hmumu/2018/bkgfit_mu_ggF_3_125.png}
%		\includegraphics[width=0.40\textwidth]{fig/hmumu/2018/bkgfit_mu_ggF_4_125.png}\\
%		\includegraphics[width=0.40\textwidth]{fig/hmumu/2018/bkgfit_mu_VBF_501_125.png}
%		\includegraphics[width=0.40\textwidth]{fig/hmumu/2018/bkgfit_mu_VBF_502_125.png}\\
%		\includegraphics[width=0.40\textwidth]{fig/hmumu/2018/bkgfit_mu_ggF_503_125.png}
%		\caption{Fits to simulated $m_{\mu^+\mu^-\gamma}$ resonant background distributions from $\PH\to\Pgmp\Pgmm$ for
%			 $m_\PH=125\GeV$ for the 2018 data-taking period.
%			 The blue line shows the total fit function, the green line shows the Crystal Ball function component, and the red line shows the Gaussian function component.
%			 The top four plots correspond to the untagged categories, and the bottom three plots correspond to the dijet categories.}
%		\label{fig:mubkgfit}
%	\end{center}
%\end{figure}
%
%\begin{figure}
%	\begin{center}
%		\includegraphics[width=0.40\textwidth]{fig/hmumu/2018/bkgfit_ele_mu_ZH_6789_125.png}
%		\includegraphics[width=0.40\textwidth]{fig/hmumu/2018/bkgfit_ele_mu_WH_6789_125.png}
%		\caption{Fits to simulated $m_{\ell^+\ell^-\gamma}$ resonant background distributions from $\PH\to\Pgmp\Pgmm$ in the electron and muon channels combined in the lepton-tagged category for
%            		 $m_\PH=125\GeV$ for the 2018 data-taking period.
%        		 The left plot shows the fit to simulated ZH production events, and the right plot shows the fit to simulated WH production events. 
%			 The blue line shows the total fit function, the green line shows the Crystal Ball function component, and the red line shows the Gaussian function component.}
%		\label{fig:elemubkgfit}
%	\end{center}
%\end{figure}


%\chapter{Title of First Appendix} 	% First appendix chapter, i.e., Appendix A.
%
%Here goes the first appendix.
%
%
%\section{First section of Appendix}	% This is appendix section 1.
%
%This is the Appendix.
%
%\subsection{First subsection of Appendix}  % This is appendix subsection 1.
%
%More text ...
%
%\subsubsection{First subsubsection of Appendix}  % This is subsubsection 1.
%
%Yet more text ...
%
%\chapter{Title of Second Appendix}
%
%In case there is more than one appendix.


%\begin{vita}                    % Vita (optional).
%
%This is the Vita. This is the Vita. This is the Vita. This is the Vita. 
%This is the Vita. This is the Vita. This is the Vita. This is the Vita. 
%This is the Vita. This is the Vita. This is the Vita. This is the Vita. 
%This is the Vita. This is the Vita. This is the Vita. This is the Vita. 
%
%This is the Vita. This is the Vita. This is the Vita. This is the Vita. 
%This is the Vita. This is the Vita. This is the Vita. This is the Vita. 
%This is the Vita. This is the Vita. This is the Vita. This is the Vita. 
%This is the Vita. This is the Vita. This is the Vita. This is the Vita. 
%
%This is the Vita. This is the Vita. This is the Vita. This is the Vita. 
%This is the Vita. This is the Vita. This is the Vita. This is the Vita. 
%This is the Vita. This is the Vita. This is the Vita. This is the Vita. 
%This is the Vita. This is the Vita. This is the Vita. This is the Vita. 
%
%
%\end{vita}


\end{document}

