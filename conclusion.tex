\chapter{Conclusion}\label{sec:conclusion}
A search is performed for a standard model (SM) Higgs boson decaying into a lepton pair ($\epem$ or $\mpmm$) and a photon with $m_{\lplm}>50$\GeV. 
The analysis is performed using a sample of proton-proton ($\Pp\Pp$) collision data at $\sqrt{s}=13$\TeV, corresponding to an integrated
luminosity of \LumiT\fbinv. 
The main contribution
to this final state is 
from Higgs boson decays to a $\PZ$ boson and a photon ($\PH\to\PZ\gamma\to\lplm\gamma$). 
The best fit value of the signal strength $\hat{\mu}$ for $m_{\PH}=\mH\GeV$ is $\hat{\mu}=\signalstrengthExpanded=\signalstrength$.  
This measurement corresponds to $\sigma(\Pp\Pp\to\PH)\brzg=\,$\br\,pb. 
The measured value is \compatibility\, standard deviations higher than the SM prediction.
The observed (expected) local significance is \obssig\,(\expsig) standard deviations, where the expected significance is determined for the SM hypothesis.
The observed (expected) upper limit at 95\% confidence level on $\mu$ is \obslimit\,(\explimit). 
In addition, a combined fit with the $\PH\to\gamma\gamma$ analysis of the same data set is performed to measure 
the ratio $\brzg/\brgg=\brRatio$, which is consistent
with the ratio of $0.69 \pm 0.04$ predicted by the SM at the \brRatioCompat\, standard deviation level. 

While no observation has yet been made of \hzg{}, the CMS Run 2 search marks the most optimal analysis carried out by 
the collaboration to date in this channel. In particular, the kinematic fit procedure and machine learning-based categorization
strategy have greatly improved the sensitivity of the analysis. Moreover, detailed studies have led to a far more robust statistical 
approach to background modeling than in the previous CMS searches, and these studies have greatly improved our understanding. We are confident that
the innovations and insights of this analysis position CMS well for an eventual discovery of \hzg{}. The wealth of data to be acquired and analyzed during
Run 3 of the LHC will provide an excellent opportunity. We wish our colleagues the best of luck as they continue the search!
