%%%%%%%%%%%%%%%%%%%%%%%%%%%%%%%%%%%%%%%%%%%%%%%%%%%%%%%%%%%%%%%%%%%%%%
% James Bueghly's Northwestern Ph.D. Thesis
% Department of Physics and Astronomy
% CMS Run II Higgs to Z+gamma Analysis
%%%%%%%%%%%%%%%%%%%%%%%%%%%%%%%%%%%%%%%%%%%%%%%%%%%%%%%%%%%%%%%%%%%%%%

%%%%%%%%%%%%%%%%%%%%%%%%%%%%%%%%%%%%%%%%%%%%%%%%%%%%%%%%%%%%%%%%%%%%%%
% template from Northwestern Department of Mathematics
% nuthesis-template.tex - Miguel A, Lerma - 4/23/2018
%                         mlerma@math.northwestern.edu
%%%%%%%%%%%%%%%%%%%%%%%%%%%%%%%%%%%%%%%%%%%%%%%%%%%%%%%%%%%%%%%%%%%%%%

% The nuthesis class is based on % amsbook.cls.
\documentclass[12pt]{template/nuthesis}	

\author{Name of Author}

\title{Title of the Dissertation}

%\degree{DOCTOR OF PHILOSOPHY}  % Default: DOCTOR OF PHILOSOPHY

%\field{Mathematics}            % Default: Mathematics

%\graduationmonth{June}         % The default is June or December
                                % depending on current date.

%\graduationyear{2003}          % Default: current year.


				% Use \includeonly to select the 
%\includeonly{chap1,chap2,...}	% chapters to include if you are 
				% using the \include command below.
				% This way you can latex only a the 
				% part you are working on, which 
				% is faster than latexing the entire 
				% thesis. 


\begin{document}
%	
%	THE BODY OF YOUR THESIS STARTS HERE
%

%%%%%%%%%%%%%%%%%%%%%%
% Some initial stuff %
%%%%%%%%%%%%%%%%%%%%%%

\frontmatter		% Preliminary pages start here.

\maketitle		% Produces the title page.

\copyrightpage		% Creates the copyright page.


\abstract		% Abstract.

This is the abstract.

\acknowledgements	% Acknowledgements (optional).

Text for acknowledgments.

\preface		% Preface (optional).

This is the preface.


%% A few more optional pages (uncomment if needed)
%
%\listofabbreviations 
%
%This is the list of abbreviations (optional).
%
%\glossary
%
%This is the glossary (optional).
%
%\nomenclature
%
%This is the nomenclature (optional).
%
%% Note that the dedication text must be passed as an argument
%% of the \dedication command
%\dedication{This is the dedication (optional).}
%

\clearpage\phantomsection % needed for the hyperlinks to work correctly
\tableofcontents	% Table of Contents will be automatically
			% generated and placed here.

\clearpage\phantomsection % needed for the hyperlinks to work correctly
\listoftables		% List of Tables and List of Figures will be placed

\clearpage\phantomsection % needed for the hyperlinks to work correctly
\listoffigures		% here, if applicable (optional).



\mainmatter             % Actual text starts here.

%%%%%%%%%%%%%%%%%%%%%%%%%%%
% Actual text starts here %
%%%%%%%%%%%%%%%%%%%%%%%%%%%

% If there is an introduction it must be the first chapter

\chapter{Introduction or Title of First Chapter}	% The first chapter.
				% \chapter command is of the form
				% \chapter[..]{..} or \chapter{..} where
	... text ...		% {chapter heading} and [entry in table of
				% contents].
\section{First section of chapter 1}
				% IMPORTANT: If your chapter heading consists
				% of more than one lines, it will be auto-
	... text ...		% matically broken into separate lines.
				% However, if you don't like the way LaTeX
				% breaks the chapter heading into lines, use
\section{Another section}	% `\newheadline' command to break lines.
				% Never use \\ in sectional (e.g., chapter,
	... text ...		% section, subsection) headings.

\chapter{Title of Second Chapter}	% Chapter 2.

	... text ...

\section{First section of chapter 2}

	... text...

\subsection{A subsection}

	... more text ...

\subsubsection{A subsubsection}

	... more text ...


% Alternatively, you may write the chapters in separate
% files, say chap1.tex, chap2.tex, etc., and include them 
% with commands:

%\include{chap1}

%\include{chap2}

% The command \includeonly above allows to include a selected 
% set of chapters only.

% Uncomment the 'singlespace' environment and '\bibsep' command
% if needed - some bibliographic styles overide the definition
% of 'thebibliography' in nuthesis.cls
%
\begin{singlespace}
%\bibsep 12pt
\clearpage\phantomsection % needed for hyperlikns to work correctly
\begin{thebibliography}{xxx}

\bibitem{label1} A bibliographic item.  A bibliographic item.  A
bibliographic item.  A bibliographic item.
\bibitem{label2} Another bibliographic item.  
\bibitem{label3} Yet another bibliographic item.  
% The usages of \bibitem and \cite{..} are 
% explained in Section 4.3 (page 73) of % LaTeX manual.
% Or you may use BibTeX.
\end{thebibliography}
\end{singlespace}

% (The following suggested by Francisco Iacobelli - 5/11/2010)
% In case, you want to use BibTeX, you should replace (or comment)
% the bibliography environment.
% Instead uncomment the following lines and replace <bib file>
% with your .bib file:
% \begin{singlespace}
% \bibsep 12pt
% \bibliographystyle{acm} %or another suitable style.
% \bibliography{<bib file>}
% \end{singlespace}


\appendix		% Appendix begins here (optional).


\chapter{Title of First Appendix} 	% First appendix chapter, i.e., Appendix A.

Here goes the first appendix.


\section{First section of Appendix}	% This is appendix section 1.

This is the Appendix.

\subsection{First subsection of Appendix}  % This is appendix subsection 1.

More text ...

\subsubsection{First subsubsection of Appendix}  % This is subsubsection 1.

Yet more text ...

\chapter{Title of Second Appendix}

In case there is more than one appendix.


\begin{vita}                    % Vita (optional).

This is the Vita. This is the Vita. This is the Vita. This is the Vita. 
This is the Vita. This is the Vita. This is the Vita. This is the Vita. 
This is the Vita. This is the Vita. This is the Vita. This is the Vita. 
This is the Vita. This is the Vita. This is the Vita. This is the Vita. 

This is the Vita. This is the Vita. This is the Vita. This is the Vita. 
This is the Vita. This is the Vita. This is the Vita. This is the Vita. 
This is the Vita. This is the Vita. This is the Vita. This is the Vita. 
This is the Vita. This is the Vita. This is the Vita. This is the Vita. 

This is the Vita. This is the Vita. This is the Vita. This is the Vita. 
This is the Vita. This is the Vita. This is the Vita. This is the Vita. 
This is the Vita. This is the Vita. This is the Vita. This is the Vita. 
This is the Vita. This is the Vita. This is the Vita. This is the Vita. 


\end{vita}


\end{document}

