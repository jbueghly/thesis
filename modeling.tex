\chapter{Signal and Background Modeling}
In each category, we carry out a shape analysis to search for a signal peak in the $m_{\ell\ell\gamma}$ spectrum.
The signal and background mass shapes are modeled using parametric functions. 

\subsection{Signal Modeling}
The signal model is defined as the sum of Crystal Ball~\cite{CB-Oreglia} and Gaussian functions.
The signal shape parameters are determined by fitting this model to simulated signal events in each category.
To account for differences in mass resolution, these fits are performed separately for the event samples used to model each data-taking year, as well as for muon and electron channel events.
This results in six signal models that are summed to give the total signal expectation in a given category.
Separate sets of parameter values are found by fitting simulated events with $m_\PH$ of $120$, $125$, and $130$\GeV.
Using linear interpolation, parameter values are also determined at 1\GeV intervals in $m_\PH$ from $120$--$130$\GeV, as well as at 125.38\GeV.
In the fit to data, the mean and resolution parameters are allowed to vary subject to constraints from several systematic uncertainties, described in Section~\ref{sec:uncertainties}, while the remaining parameters are held fixed. 
