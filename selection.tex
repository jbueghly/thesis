\chapter{Object and Event Selection}

\section{Triggers}
The topology and basic kinematics of the \hzg process guide the choice of triggers used in this analysis. As this is a three body
decay, the \pT of the photon tends to be less than in other channels like \hgg. Consequently, the CMS photon trigger \pT thresholds 
are too great to make them viable options. Instead, we trigger on the leptons arising from the decay of the Z boson, which tend 
to have larger \pT. The best approach to maximize signal efficiency is to use the double lepton triggers. The double muon trigger 
has \pT thresholds of 17 and 8 GeV, and the double electron trigger has thresholds of 23 and 12 GeV. These are the lowest 
unprescaled double lepton triggers generally available for CMS Run 2 analysis. Events passing both the double muon and 
double electron triggers are treated as double muon events.

The triggers are applied to both data and simulation. Trigger efficiencies and scale factors are measured using the 
simulation samples corresponding to each data-taking year. These measurements use a tag and probe method [CITE]. The tag and probe  method
takes advantage of the high purity of $Z \rightarrow \ell^{+}\ell{-}$ events near the Z mass peak. One lepton functions as the 
tag, and satisfies a set of tight trigger, identification, isolation, \pT requirements. The second lepton, the probe, must pass a looser selection and is used to measure the efficiency in question. 
Using this approach, trigger efficiencies for each leg of a given 
double lepton trigger are measured in both data and simulation. Then a corrective scale factor, defined as the ratio of data efficiency 
to simulation efficiency, is applied to the simulation. Scale factors are measured and applied in bins of \pT and $|\eta|$.

For the double electron trigger efficiency measurements, the tag electron must satisfy the following requirements. It must pass 
the single electron trigger, pass a tight cut-based identification, have \pT $> 30 \, (35)$ in 2016 (2017/2018), and have $|\eta| < 2.5$. 
The probe electron must pass the loose electron MVA identification (with isolation) requirement. The efficiencies for each leg 
of the trigger are measured separately, so in each case, the probe electron must match the trigger leg being measured. The efficiencies 
and scale factors of each double electron trigger leg for 2016, 2017, and 2018 are shown in Figures [FIGS]. 

For the double muon trigger efficiency measurements, the tag muon must satisfy the following requirements. It must pass the single 
muon trigger, pass a tight cut-based identification, have \pT $> 26 \, (29)$ GeV for 2016 (2017/2018) and satisfy $|\eta| < 2.4$. The 
probe muon must pass the $H \rightarrow ZZ$ identification and isolation cuts. The details of the $H \rightarrow ZZ$ muon 
identification will be described later. The efficiencies for each leg of the trigger are measured separately, so in each case, the probe 
muon must match the trigger leg being measured. The efficiencies and scale factors of each double muon trigger leg for 2016, 2017, 
and 2018 are shown in Figures [FIGS]. 

\section{Muon Selection}
A loose cut-based muon identification is used in the analysis. This identification was originally developed by the \hzz 
analysis [REF] and is well-suited for \hzg due to the similarity of the multiple, potentially soft, muons in the final state. 
All muons are first required to pass a set of common cuts, followed by a separate set of cuts for high \pT (greater than 200 GeV) 
and low \pT muons. All muons must satisfy \pT $>$ 5 GeV, $|\eta| < 2.4$, $|d_{xy}| < 0.5$ cm, and $|d_{z}| < 1$ cm, where $d_{xy}$ and 
$d_{z}$ are impact parameters defined with respect to the primary vertex of the interaction using the best muon track. 
Additionally, the three dimensional impact parameter (analogously defined) must have a magnitude less than four times its uncertainty.
Muons must either be reconstructed as global muons or tracker muons. Muons with standalone tracks (tracks only in the muon system) are
rejected. Muons must pass a particle flow-based isolation requirement, where the relative particle flow isolation 
within a $\Delta R = 0.3$ cone is defined as 
\begin{equation}
\label{eqn:pfiso}
	\mathcal{I} \equiv \Big( \sum p_{T}^\text{charged} + \max\big[ 0, \sum p_{T}^\text{neutral} +\sum p_{T}^{\gamma} - p_{T}^\mathrm{PU}(\ell) \big] \Big) / p_{T}^{\ell}.
\end{equation}
Muons must satisfy $\mathcal{I} < 0.35$. 

For muons with \pT $<$ 200 GeV, muons satisfying the common requirements and identified by the particle flow identification
algorithm are selected. For muons with \pT $>$ 200 GeV, muons are selected if they pass the particle flow identification or if they pass
a set of high-$p_{T}$ requirements. These requirements are the following: the muon must be matched to segments in at least two muon 
stations; satisfy $\frac{p_{T}}{\sigma_{p_{T}}} < 0.3$, $|d_{xy}| < 0.2$ cm, $|d_{z}| < 0.5$ cm, have at least one pixel hit, and have 
tracker hits in at least six tracker layers.

In 2016, data-taking was affected by a problem in which the level one trigger sent only one candidate per 60$^{\circ}$ sector instead 
of up to three [REF?]. As a result, when two muons in the same endcap had a low $\Delta \phi$ separation, only one would fire the 
trigger. To account for this, 2016 data events containing identified muons with $\Delta \phi < 70^{\circ}$ in the same endcap region 
are rejected. 

Identification efficiencies and scale factors corresponding to the muon identification above are measured and provided by the 
\hzz analysis working group, and are shown in Figure [FIG].  

\section{Electron Selection}
Electrons are identified using a boosted decision tree multivariate (MVA) discriminator trained on Drell-Yan plus jets simulation 
with prompt electrons matched to generator-level objects as signal and unmatched and non-prompt electrons as background. The features 
used in the training include \pT, supercluster $\eta$, shower shape variables, ratio of hadronic to electromagnetic energy, track and 
pixel hit variables, and isolation variables [REF]. These features are sensitive to bremsstrahlung along the electron trajectory, 
momentum-energy matching between electron trajectory and ECAL cluster, shower shape, and electrons from photon conversions.
Since isolation features are included in the training of the discriminator, there is no need for separate isolation cuts. 
Electrons pass the identification requirement of the \hzg analysis if their discriminator score is higher than a loose working point 
value, corresponding to 98\% signal efficiency. In addition to the MVA cut, electrons must have $|d_{xy}| < 0.5$ cm and $|d_{z}| < 1$ cm
with respect to the primary vertex. Electrons with \pT $\leq$ 7 GeV are rejected. 

Electron identification efficiencies and scale factors are measured using a tag and probe method on dielectron events near the 
Z boson peak. These electrons must pass the single electron trigger with a \pT threshold of 27 (32) GeV for 2016 (2017/2018) data. 
The dielectron mass must be between 60 and 120 GeV. The tag electron must pass a tight cut-based electron identification and have 
\pT $>$ 30 (35) GeV for 2016 (2017) and $|\eta|$ < 2.5. The probe electron must pass the loose MVA identification cut and associated 
impact parameter and \pT cuts described above. Identification efficiencies and scale factors are measured and applied in bins of 
\pT and supercluster $\eta$. The scale factors for each data-taking year are shown in Figures [FIGS]. 

\section{Photon Selection}
Photons are identified using a boosted decision tree MVA discriminator trained on photon plus jets simulation. The features used in the 
training include supercluster kinematics, isolation variables, and shower shape variables [REF]. To reject electrons faking photons, 
a conversion-safe electron veto is applied. In order to improve agreement between simulation and data in the \hzg analysis, shower 
shape corrections are taken from the Higgs to diphoton analysis [REF] and applied to simulated events. The original MVA provided by the 
EGamma POG is then reevaluated after these corrections. The following features are corrected: $R_{9}$, defined as the ratio of energy in
the 5x5 array of ECAL crystals to the supercluster energy; $S_{4}$, defined as the ratio of the maximum energy 2x2 array to the energy of 
the 5x5 array; the energy weighted shower widths $\sigma_{\eta}$ and $\sigma_{\phi}$; the energy weighted widths by crystal index 
$\sigma_{i\eta i\eta}$ and $\sigma_{i\eta i\phi}$; photon isolation, charged isolation with respect to the primary vertex, 
and charged isolation with respect to the worst vertex choice. The validity of the shower shape corrections is checked using tag 
and probe procedures for $Z\rightarrow e^{+}e^{-}$ events where the probe electron mimics a photon and $Z\rightarrow \mu^{+}\mu^{-}$
events with an FSR photon. A comparison of the agreement between uncorrected and corrected simulation with data is shown in 
Figures [FIGS]. Simulation comparisons with data for individual shower shape features before and after correction can be found 
in Appendix [APPENDIX], Figures [FIGS]. 

After deriving the corrected photon MVA discriminator, 90\% signal efficiency working point cuts for barrel and endcap photons 
are determined for the \hzg analysis. The working points are defined based on real photons from SM $Z\gamma$ simulation, and correspond 
to discriminator scores greater than -0.4 (-0.59) for barrel (endcap) photons. A comparison of these working points with the standard 
EGamma POG 90\% efficiency working points, plotted on the receiver operator characteristic (ROC) curve for SM $Z\gamma$ and Z plus jets 
simulation, is shown in Figure [FIG]. The efficiency of the photon identification is measured with $Z\rightarrow e^{+}e^{-}$ data
using a tag and probe technique. The tag electron must pass the single electron trigger with \pT threshold 27 (32) GeV in 
2016 (2017/2018), tight cut-based identification, have \pT $>$ 30 (35) GeV for 2016 (2017/2018), and have $|\eta| < 2.5$. The 
probe electron must pass the photon MVA identification with shower shape corrections described above. The scale factors, measured and 
applied in bins of \pT and supercluster $\eta$, are shown in Figure [FIG]. The efficiencies and scale factors for the conversion-safe 
electron veto are measured by the EGamma POG using $Z\mu^{+}\mu^{-}$ events with an FSR photon and applied in the \hzg analysis. 
The scale factors are cataloged in Table [TAB]. 

\section{Jet Selection}
Jets are selected in order to categorize events coming from potential VBF Higgs production, but no jet multiplicity requirement is present 
for the \hzg selection. In fact, the majority of simulation and data events selected in the analysis have no jets. However, identifying 
and selecting jets to categorize VBF events can still significantly improve the sensitivity of the search. Jets are required to pass 
a loose cut-based identification in 2016 and a tight cut-based identification in 2017 and 2018. These sets of identification cuts are
determined and provided by the JetMET POG. Additionally, jets must satisfy \pT $>$ 30 GeV, $|\eta| < 4.7$, and $\Delta R > 0.4$ with 
respect to each lepton and the photon selected in the analysis. An issue with noise in the ECAL endcap in 2017 caused an artificial 
increase in jet multiplicity in data within a specific kinematic phase space [REF]. To mitigate this, jets are rejected if they have 
raw \pT $<$ 50 GeV and $2.65 < |\eta| < 3.139$. This cut reduces the efficiency to reconstruct dijet pairs by 12\% in the specified 
region. To tag VBF events, we are interested in dijet pairs. To this end, if there are more than two jets satisfying the above criteria, 
only the two jets with highest \pT are selected. 

\section{Object Corrections}

