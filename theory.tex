\chapter{Theory}

\section{The Standard Model}

\section{Electroweak Theory}

%\section{Quantum Chromodynamics}

\section{Spontaneous Symmetry Breaking (Higgs Mechanism)}
The electroweak Lagrangian lacking the Higgs term \textcolor{red}{CITE EQUATION} is insufficient. In particular, it does not provide
an explanation for massive gauge bosons. Experimental evidence for massive gauge bosons dates back to the \textcolor{red}{early?}
twentieth century. Beta decay experiments indicated charged current interactions mediated by a massive W boson, and later experiments 
indicated additional neutral current interactions mediated by a massive Z boson (\textcolor{red}{CHECK THIS}). Later in the twentieth 
century (\textcolor{red}{GIVE SPECIFIC DATES/REFS}) the W and Z bosons were discovered. Indeed, the W and Z masses have been found 
to be roughly 80 GeV and 91 GeV respectively. An explanation came in the form of the Higgs mechanism (\textcolor{red}{CITE}),
which spontaneously breaks the SU(2)xU(1) gauge symmetry of the elecroweak Lagrangian. In addition, there arises a real 
Higgs field accompanied by a massive Higgs boson, and fermion masses are explained via their Yukawa couplings to the Higgs boson. 
The discovery of the Higgs boson and the subsequent measurements of its properties have provided proof that the Higgs mechanism 
is indeed a central piece of the Standard Model. 

To understand how the Higgs mechanism works, consider the introduction of a complex scalar field $\sf \Phi$, which 
transforms as a doublet under SU(2).
\begin{equation}
    \label{eqn:higgsField}
    \Phi = 
    \begin{bmatrix}
        \phi^{+} \\ 
        \phi^{0}
    \end{bmatrix}
\end{equation}
Its contribution to the Lagrangian is given by:
\begin{equation}
    \mathcal{L_{H}} = (D^{\mu}\Phi)^{\dagger}(D_{\mu}\Phi) - V(\Phi)
    \label{LHiggs}
\end{equation}
where the Higgs potential takes the form
\begin{equation}
    V(\Phi) = \mu^{2}|\Phi^{\dagger}\Phi| + \lambda \Big(|\Phi^{\dagger}\Phi|\Big)^{2}.
    \label{VHiggs}
\end{equation}
$D_{\mu}$ is the covariant derivative
\begin{equation}
    D_{\mu} = \partial_{\mu} - \frac{ig}{2}\tau\cdot A_{\mu} - \frac{ig'}{2}B_{\mu}Y.
    \label{EWCovariantDeriv}
\end{equation}
Consider the case that the parameters of the Higgs potential $\lambda$ and $\mu$ satisfy the conditions 
$\lambda > 0$ and $\mu^{2} < 0$. Then the shape of the potential is shown in \ref{fig:VHiggs}. Clearly, there is no minimum of the 
potential at $\Phi = 0$. Rather, an infinite set of minima lie around a circle in the complex plane. Hence, it is said that $\Phi$ 
has a nonzero vacuum expectation value (VEV). The value of the VEV in terms of $\mu$ and $\lambda$ can be determined by explicitly
minimizing the potential:
\begin{align}
    \frac{\partial}{\partial(\Phi^{\dagger}\Phi)}V(\Phi) &= 0 \\
    \mu^{2} + 2\lambda\Big(|\Phi^{\dagger}\Phi|\Big) &= 0 \\
    \mu^{2} + 2\lambda\Big[(\phi^{+})^{2} + (\phi^{0})^{2}\Big] &= 0 
    \label{VHiggsMinimization}
\end{align}
Clearly, this can be minimized in many ways depending on individual values of $\phi^{+}$ and $\phi^{0}$ in the vacuum. By convention,
and without generality, we choose the case in which $\phi^{+} = 0$. In this case we obtain the equation
\begin{equation}
    \phi^{0} = \sqrt{\frac{-\mu^{2}}{2\lambda}} = \frac{1}{\sqrt{2}}v
\end{equation}
where we have defined $v \equiv \sqrt{-\mu^{2}/\lambda}$.

The existence of the Higgs VEV leads naturally to the masses of the W and Z bosons. This can be shown by squaring the covariant 
derivative acting on the scalar doublet $\Phi$ and evaluating the result in the vacuum. Note that in the vacuum, all terms involving 
the partial derivative $\partial_{\mu}$ will yield zero contribution. Therefore, keeping only the relevant terms, the squared covariant 
derivative reduces to
\begin{equation}
    |D_{\mu}|^{2} \rightarrow (\frac{g}{2}A_{\mu}^{a}\tau^{a} +\frac{g'}{2}B_{\mu})(\frac{g}{2}A^{b\mu}\tau^{b} + \frac{g'}{2}B^{\mu})
    \label{covDerivSquared}
\end{equation}
Evaluating this in the vacuum yields
\begin{align}
    \Delta\mathcal{L} &= \frac{1}{2}
    \begin{bmatrix}0 & v
    \end{bmatrix}
    (\frac{g}{2}A_{\mu}^{a}\tau^{a} +\frac{g'}{2}B_{\mu})(\frac{g}{2}A^{b\mu}\tau^{b} + \frac{g'}{2}B^{\mu}) 
    \begin{bmatrix}
        0 \\ 
        v
    \end{bmatrix} \\ 
    \Delta\mathcal{L} &=
    \frac{1}{2}\frac{v^{2}}{4}[g^{2}(A_{\mu}^{1})^{2} g^{2}(A_{\mu}^{2})^{2} + (g'B_{\mu} - gA_{\mu}^{3})^{2}]
\end{align}
From this, we can identify the fields and masses for the positively and negatively charged W bosons and the neutral Z boson and photon. 
These are as follows:
\begin{align}
    W_{\mu}^{\pm} &= \frac{1}{\sqrt{2}}(A_{\mu}^{1} \mp iA_{\mu}^{2}) \; &m_{W} = \frac{gv}{2} \\
    Z_{\mu}^{0} &= \frac{1}{\sqrt{g^{2} + g'^{2}}}(gA_{\mu}^{3} - g'B_{\mu}) \; &m_{Z} = \sqrt{g^{2} + g'^{2}}\frac{v}{2} \\
    A_{\mu} &= \frac{1}{\sqrt{g^{2} + g'^{2}}}(g'A_{\mu}^{3} + gB_{\mu}) \; &m_{A} = 0
    \label{gaugeBosonsAndMasses}
\end{align}

\section{Higgs Production}

\section{Higgs Decay}

\section{Physics Beyond the Standard Model}

