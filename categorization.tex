\chapter{Event Categorization}

The sensitivity of the \hzg search can be significantly improved by leveraging the differences between signal events
arising from different Higgs boson production modes with different final state topologies and kinematics. 
This is achieved by defining mutually exclusive categories targeting these different types of signal events. In previously 
published CMS searches for \hzg [REFS], cut-based approaches were used to define the categories. Broadly, the presence of at least 
one additional lepton was used to tag VH and ttH production, a dijet system was used to tag VBF production, a boosted jet was 
used to tag gluon-gluon fusion events recoiling off of a jet, and properties of the photon were used to divide the remaining 
events into kinematically distinct untagged categories. The specific definitions of the categories used in the 2016 search 
at 13 TeV are shown in Figure [FIG]. 

The categorization of the present search is inspired by the previous CMS searches, but is significantly more sophisticated and 
better optimized than those searches. The lepton tag category remains, as do the concepts of dijet and untagged categories. 
However, the dijet and untagged categories are determined via the training of two BDTs, with category 
boundaries optimized using the resulting BDT scores. Moreover, the boosted category is studied and found to yield no 
improvement over the categories determined by the BDTs, so the boosted category is dropped in the present analysis. We proceed with 
a description of the BDT procedures for the untagged and dijet cases. Then, with our BDTs in hand, we describe the procedure 
for defining an optimal set of categories to be used in the analysis. We show that this categorization procedure is more 
optimal than what has been done in past CMS searches. 

\section{Kinematic BDT}

The kinematic BDT is used to separate \hzg signal events from background events. Its primary purpose is to define the 
untagged categories, which are distinguished primarily by kinematics and the quality of the final state photon. Its secondary 
use is as an input to the dijet BDT (described below) in order to increase signal to background discrimination in the 
dijet BDT training. The kinematic BDT is trained on \hzg signal events from all Higgs production modes and background events from 
SM $Z\gamma$, Z+jets, and $t\bar{t}$. The training is restricted to use half of the simulated events, with the other half 
reserved for category optimization (described later). All training events are required to pass the basic object and 
event selection described in Chapter 6. Events from 2016, 2017, and 2018 simulation samples in both the muon and electron channels 
are combined for the training. These samples are weighted by their respective cross sections and weighted by the luminosity of
each year. 

\section{Dijet BDT}

\section{Categorization Procedure}

\section{Dropping of Boosted Category}

\section{Comparison to Previous Approach}
