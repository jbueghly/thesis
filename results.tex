\chapter{Results and Interpretation}\label{sec:results}

Figures~\ref{fig:3} and \ref{fig:4} show the $m_{\ell^+\ell^-\gamma}$ distributions of the data events in each category.
The expected SM $\PH\to\PZ\gamma$ distributions, scaled by a factor of 10, are also shown.
Figure~\ref{fig:SignalBackground} shows the signal-plus-background fit to the data and the corresponding distribution after background subtraction for the sum of all categories. 
Each category is weighted by the factor $S/(S+B)$, where $S$ is the signal yield  
and $B$ is the background yield in the narrowest mass interval containing 95\% of the signal distribution.

\begin{figure}
  \centering
  \includegraphics[width=0.45\textwidth]{fig/results/Figure_008.pdf}
  \includegraphics[width=0.45\textwidth]{fig/results/Figure_004-a.pdf}\\
  \includegraphics[width=0.45\textwidth]{fig/results/Figure_004-b.pdf}
  \includegraphics[width=0.45\textwidth]{fig/results/Figure_004-c.pdf}\\
   \caption{Fits to the $m_{\ell^+\ell^-\gamma}$ data distribution
    in the lepton-tagged (upper left), dijet 1 (upper right), dijet 2 (lower left), and
  dijet 3 (lower right) categories.
  In the upper panel, the red solid line shows the result of a signal-plus-background fit to the given category.
  The red dashed line shows the background component of the fit.
  The green and yellow bands represent the $68$ and $95$\% \CL\ uncertainties in the fit.
  Also plotted is the expected SM signal, scaled by a factor of 10.
  In the lower panel, the data minus the background component of the fit is shown. \label{fig:3}}
  \end{figure}

  \begin{figure}
  \centering
  \includegraphics[width=0.45\textwidth]{fig/results/Figure_006-a.pdf}
  \includegraphics[width=0.45\textwidth]{fig/results/Figure_006-b.pdf}\\
  \includegraphics[width=0.45\textwidth]{fig/results/Figure_006-c.pdf}
  \includegraphics[width=0.45\textwidth]{fig/results/Figure_006-d.pdf}
   \caption{Fits to the $m_{\ell^+\ell^-\gamma}$ data distribution
    in the untagged 1 (upper left), untagged 2 (upper right), untagged 3 (lower left), and
  untagged 4 (lower right) categories.
  In the upper panel, the red solid line shows the result of a signal-plus-background fit to the given category.
  The red dashed line shows the background component of the fit.
  The green and yellow bands represent the $68$ and $95$\% \CL\ uncertainties in the fit.
  Also plotted is the expected SM signal, scaled by a factor of 10.
  In the lower panel, the data minus the background component of the fit is shown. \label{fig:4}}
  \end{figure}

\begin{figure}
\centering
\includegraphics[width=0.5\textwidth]{fig/results/Figure_012.pdf}
 \caption{Sum over all categories of the data points and signal-plus-background model after the simultaneous fit to each $m_{\ell^+\ell^-\gamma}$ distribution. 
 The contribution from each category is weighted by $S/(S+B)$, as defined in the text. 
 In the upper panel, the red solid line shows the signal-plus-background fit. The red dashed line shows the background component of the fit. The green and yellow bands represent the $68$ and $95$\% CL uncertainties in the fit. Also plotted is the expected SM signal weighted by $S/(S+B)$ and scaled by a factor of 10. In the lower panel, the data minus the background component of the fit is shown.
   }
\label{fig:SignalBackground}
\end{figure}

The best fit value of the signal strength is $\signalstrengthExpanded$ at $m_\PH=\mH$\GeV.
The corresponding measured value of $\sigma(\Pp\Pp\to\PH)\mathcal{B}(\PH\to\PZ\gamma)$ is $\brExpanded$\,pb. This measurement is consistent with the SM prediction of $0.09 \pm 0.01$\,pb at the \compatibility\, standard deviation level.
Figure~\ref{fig:lim-combo125} shows the signal strengths obtained for each category separately, corresponding to the fit results shown in Figs.~\ref{fig:3} and \ref{fig:4}, as well as from simultaneous fits to the dijet categories, the untagged categories, and all categories combined. 
Among the eight categories, dijet 1 is the most sensitive.
A category compatibility $\textit{p}$-value, under the hypothesis of a common signal strength in all categories, is calculated from the likelihood ratio between the 
nominal combined fit, in which all categories have the same signal strength parameter, 
and a separate fit, in which each category has its own signal strength parameter. 
This $\textit{p}$-value is found to be \channelcompatp, corresponding to \channelcompatsigma\, standard deviations, and is driven by the dijet 3 category, which has a signal strength of $\hat{\mu}=\signalstrengthdijetthree$. 
The observed (expected) local significance is \obssig\,(\expsig) standard deviations. 
Upper limits on $\mu$ are calculated at 1\GeV intervals 
in the mass range of $120 < m_{\ell^+\ell^-\gamma} < 130\GeV$ and at $m_\PH=\mH\GeV$, as shown in Fig.~\ref{fig:lim}.
The observed (expected) limit at 95\% CL relative to the SM expectation for $m_\PH=\mH\GeV$ is $\obslimit$ ($\explimit$). 

\begin{figure}
  \centering
  %  \includegraphics[width=0.75\textwidth]{Figure_010.pdf}
   \includegraphics[width=0.7\textwidth]{fig/results/Figure_011.pdf}
    \caption{
Observed signal strength ($\mu$) for a SM Higgs boson with $m_\PH=\mH\GeV$. 
The labels ``untagged combined," ``dijet combined," and ``combined" represent the results obtained from simultaneous fits of the untagged categories, dijet categories, and full set of categories, respectively. 
%Fits perform simultaneously for the combined categories, where the untagged, dijet and all categories are considered. 
The black solid line shows $\mu=1$, and the red dashed line shows the best fit value $\hat{\mu}=\,$\signalstrength ~of all categories combined.
    \label{fig:lim-combo125}}
\end{figure}

\begin{figure}
  \centering
  \includegraphics[width=0.7\textwidth]{fig/results/Figure_009.pdf}
    \caption{
	    Upper limit ($95$\%~\CL) on the signal strength ($\mu$) relative to the SM prediction, as a function of the assumed value of the Higgs boson mass used in the fit.
    \label{fig:lim}}
\end{figure}

\section{Ratio Measurement}
The ratio $\mathcal{B}(H\rightarrow\PZ\gamma)/\mathcal{B}(H\rightarrow\gamma\gamma)$ is theoretically interesting, as it is 
potentially sensitive to beyond BSM physics, such as supersymmetry and extended Higgs 
sectors~\cite{Djouadi:1996yq,Zg_theory_extension,Zg_theory_decaywidth}.
BSM effects from models such as these typically shift the $H\rightarrow\PZ\gamma$ and $\PH\to\PGg\PGg$ branching fractions 
by different amounts, making the ratio the most sensitive observable. 
By combining the $H\rightarrow\PZ\gamma$ analysis with the existing CMS $\PH\to\PGg\PGg$ analysis, described in Ref.~\cite{CMS:2021kom}, 
we can obtain a measurement of this ratio and compare it to the SM prediction of $0.69 \pm 0.04$. An experimental advantage 
of this measurement is the cancellation of common theoretical uncertainties related to Higgs boson production and common experimental 
uncertainty sources like the L1 prefiring issue.

The ratio measurement is obtained via a combined fit of the two analyses with two parameters of interest: the $\PH\to\PGg\PGg$ signal 
strength with respect to the SM expectation ($\mu_{\gamma\gamma}$) and the ratio of the signal strengths for the two channels
($\mu_{Z\gamma} / \mu_{\gamma\gamma}$). After obtaining the best fit values for these parameters of interest, 
the ratio of signal strengths can be scaled to obtain the ratio of branching fractions. Common systematic uncertainties in the two 
analyses are correlated in the combined fit, and are listed in Table \ref{tab:corr_unc}. The combination is performed at 
a Higgs boson mass of 125.38 GeV, and discrete profiling of the background models is enabled for both analyses. 

The profile likelihood scans for the parameters of interest are shown in Figure \ref{fig:scan_br}. The best fit values of the 
parameters of interest are $\mu_{\gamma\gamma} = 1.121^{+0.095}_{-0.090}$ and 
$\mu_{Z\gamma}/\mu_{\gamma\gamma} = 2.225^{+0.926}_{-0.825}$. 
The value of $\mu_{\gamma\gamma}$ agrees well with the standalone $\PH\to\PGg\PGg$ fit. 
Multiplying through, the value of $\mu_{Z\gamma}$ is 2.49, which 
agrees well with the result of the standalone $H\rightarrow\PZ\gamma$ fit.
The measured value of $\mathcal{B}(\PH\to\PZ\gamma)/\mathcal{B}(\PH\to\gamma\gamma)$ from the combined fit with the $\PH\to\PGg\PGg$
analysis is \brRatio. This measurement is consistent with the SM prediction for the ratio at the \brRatioCompat\, standard deviation level. 
The nuisance parameter impacts on the signal strength ratio are shown in Figure \ref{fig:ratio_impacts}.

\begin{table}
   \centering   
   % \resizebox{\textwidth}{15mm}{
      
\begin{tabular}{cc}
   Source &\\\hline
   Theoretical uncertainties&\\\\
   Scale uncertainties  &  QCD scale ZH\\
                        &  QCD scale WH\\
                        &  QCD scale ggH\\
                        &  QCD scale qqH\\
                        &  QCD scale ttH\\
   PDF uncertainties    &  pdf\_Higgs\_ggH\\
                        &  pdf\_Higgs\_VH\\
                        &  pdf\_Higgs\_qqH\\
                        &  pdf\_Higgs\_ttH\\\hline
   Systematic uncertainties&\\\\
   Luminosity  &lumi\_13TeV\_Uncorrelated\_2016,2017,2018\\
                            &lumi\_13TeV\_Beam\_Current\_Calibration\\
                            &lumi\_13TeV\_Beam\_Beam\_Deflection\\
                            &lumi\_13TeV\_Dynamic\_Beta\\
                            &lumi\_13TeV\_Ghosts\_And\_Satellite\\
                            &lumi\_13TeV\_Length\_Scale\\
                            &lumi\_13TeV\_X\_Y\_Factorization\\
   Parton showering        &PartonShower\_norm\\
   Underlying events        &UnderlyingEvent\_norm\\
   L1-prefire           &2016,2017 L1 prefire uncertainty\\
                        

\end{tabular}
% }
\caption{Correlated uncertainties between $\PH\to\cPZ\gamma$ and $\PH\to\gamma\gamma$ for the $\mathcal{B}(\PH\to\cPZ\gamma)/\mathcal{B}(\PH\to\gamma\gamma)$ measurement.}
\label{tab:corr_unc}
\end{table}

\begin{figure}
   \begin{center}
   \includegraphics[width=0.45\textwidth]{fig/results/ratio/scan_mu_BR_gamgam.pdf}
   \includegraphics[width=0.45\textwidth]{fig/results/ratio/scan_mu_BR_Zgam_r_BR_gamgam.pdf}\\
   \caption{Profile likelihood scans of $\mu_{\gamma\gamma}$ and $\mu_{\PZ\gamma}/\mu_{\gamma\gamma}$ with fixed $m_H=125.38\GeV$.}
   \label{fig:scan_br}
   \end{center}    
\end{figure}
