\chapter{Analysis Overview}\label{sec:analysis_overview}

\section{History of the \hzg{} Search}
Both the CMS and ATLAS collaborations have undertaken the search for \hzg{} since Run 1 of the LHC. 
Before the CMS Run 2 analysis, which is the focus of this thesis, each collaboration published results using 
Run 1 data at $\sqrt s = 7$ and 8\TeV~\cite{cms-HZG,atl-HZG}, CMS published a result with 2016 data at $\sqrt s = 13\TeV$~\cite{Sirunyan:2018tbk}, and 
ATLAS published a full Run 2 result at $\sqrt s = 13\TeV$~\cite{Aad:2020plj}. As such, the present analysis represents the continuation 
of a broader effort, so we will give a brief history of these prior searches to put our work in context. While 
some of the analysis strategies of past searches inform our current strategy, there are also many differences 
and innovations in our analysis. We will emphasize the ways in which our analysis 
overlaps and differs with previous approaches, and will show that our innovations have significantly 
improved the expected sensitivity and statistical robustness of the search. 

\subsection{The Run 1 Searches}
In 2014, ATLAS published its Run 1 results~\cite{atl-HZG} for the search for \hzg{} where $\mathrm{Z}\to\ell^+\ell^-$ with $\ell=\mathrm{e}$ or $\mu$. The baseline strategy was a standard dilepton 
plus photon selection, with $m_{\ell^+\ell^-}$ near the Z boson mass and $\ell^+\ell^-\gamma$ invariant mass ($m_{\ell^{+}\ell^{-}\gamma}$) between 115 and 170\GeV. The resolution of $m_{\ell^{+}\ell^{-}\gamma}$ was improved using a kinematic fit procedure, accounting for the true Z boson line shape. Events were categorized based on lepton flavor; the pseudorapidity difference between the photon and Z boson; and the variable $p_{\mathrm{T}}^{t}$, defined as $\abs{\vec{p}_\mathrm{T}^{\,\PZ\gamma}\times\hat{t}}$, where $\hat{t}=(\vec{p}_\mathrm{T}^{\,Z}-\vec{p}_\mathrm{T}^{\,\gamma})/\abs{\vec{p}_\mathrm{T}^{\,Z}-\vec{p}_\mathrm{T}^{\,\gamma}}$~\cite{Ackerstaffetal.1998,VESTERINEN2009432}, the $\pt$ of the $\PZ\gamma$ system that is perpendicular to the difference of the three-momenta of the $\PZ$ boson and the photon, a quantity that is strongly correlated with the \pt of the $\lplm\gamma$ system. Limits on the signal strength relative to the SM prediction, shown in Fig. \ref{fig:run1_limits} (left), were obtained from a simultaneous fit to the $m_{\ell^{+}\ell^{-}\gamma}$ distributions in all categories.

The CMS Run 1 analysis~\cite{cms-HZG} was published in 2015 for the $\mpmm\gamma$ and $\epem\gamma$ final states. It employed a standard dilepton plus photon selection, but the $m_{\ell^+\ell^-\gamma}$ range was 100--190\GeV, which includes a kinematic turn-on around 110--115\GeV, mainly driven by the photon \pt selection. The choice to model this turn-on increased the complexity of the fit relative to the ATLAS analysis above, but was made in order to minimize potential bias. Events were categorized based on flavor; presence of a dijet system; lepton and photon pseudorapidity; and the photon $\mathrm{R}_9$, which is the energy sum of the 3x3 ECAL crystal array centered on the most energetic crystal divided by the supercluster energy, and is associated with the quality of the photon. The CMS Run 1 limits on the signal strength are shown in Fig. \ref{fig:run1_limits} (right). 

\begin{figure}[tb]
  \centering
   \includegraphics[width=0.45\textwidth,height=0.33\textwidth]{fig/overview/atl_run1_lim.png}
   \includegraphics[width=0.45\textwidth,height=0.33\textwidth]{fig/overview/cms_run1_lim.png}
	\caption
	[ATLAS (left) and CMS (right) Run 1 limit results as a function of $m_\PH$.]
	{ATLAS (left~\cite{atl-HZG}) and CMS (right~\cite{cms-HZG}) Run 1 limit results as a function of $m_\PH$.}
	\label{fig:run1_limits}
\end{figure}


\subsection{The Previous Run 2 Searches}
The first CMS \hzg{} results at $\sqrt s = 13\TeV$~\cite{Sirunyan:2018tbk} were published in 2018 using the 2016 data set, corresponding to an integrated luminosity of 35.9\fbinv. The search strategy was largely similar to the CMS Run 1 analysis, with a few main differences. First, a boosted category was introduced to the analysis, defined by $\pt^{\lplm\gamma}>60\GeV$. This category targeted events with a Higgs boson recoiling against a jet. Second, the range of $m_{\lplm\gamma}$ was restricted to 115--170\GeV in order to avoid fitting the kinematic turn-on. The decision of whether or not to fit the turn-on became critically important in the full CMS Run 2 analysis, and will be discussed in more detail later in this thesis. Limits from the CMS 2016 search are shown in Fig. \ref{fig:run2_prev_results} (left). 

The ATLAS full Run 2 search~\cite{Aad:2020plj} was published in 2020, with a largely similar strategy to the Run 1 analysis. As in Run 1, a kinematic fit was used to improve the $m_{\lplm\gamma}$ resolution and the variable $p_{\mathrm{T}}^{t}$ was used for categorization. However, there were a few major differences introduced in the Run 2 analysis. The fit range was changed to $105\GeV < m_{\lplm\gamma} < 160\GeV$ and some aspects of the categorization strategy were updated. For categorization, the most significant change was the use of a boosted decision tree (BDT) trained to separate VBF Higgs production events from other Higgs production mechanisms and backgrounds. As will be discussed later in this thesis, the use of BDTs for categorization can significantly improve the sensitivity of searches for \hzg{}. Figure \ref{fig:run2_prev_results} (right) shows a summary of the fit results for the ATLAS Run 2 search. An excess in data was observed, corresponding to $2.2$ standard deviations for $m_{\PH}=125.09\GeV$.

\begin{figure}[tb]
  \centering
   \includegraphics[width=0.45\textwidth,height=0.33\textwidth]{fig/overview/cms_2016_lim.png}
   \includegraphics[width=0.45\textwidth,height=0.33\textwidth]{fig/overview/atlas_run2_fit.png}
	\caption
	[Left: CMS 2016 limit results as a function of $m_\PH$. Right: ATLAS Run 2 signal plus background fit, for $m_{\PH}=125.09\GeV$, weighted by ln(1+$\mathrm{S}_{68}/\mathrm{B}_{68})$, where $\mathrm{S}_{68}$ and $\mathrm{B}_{68}$ are the expected signal and background events in an $m_{\lplm\gamma}$ window containing 68\% of the expected signal yield.]
	{Left: CMS 2016 limit results~\cite{Sirunyan:2018tbk} as a function of $m_\PH$. Right: ATLAS Run 2 signal plus background fit~\cite{Aad:2020plj}, for $m_{\PH}=125.09\GeV$, weighted by ln(1+$\mathrm{S}_{68}/\mathrm{B}_{68})$, where $\mathrm{S}_{68}$ and $\mathrm{B}_{68}$ are the expected signal and background events in an $m_{\lplm\gamma}$ window containing 68\% of the expected signal yield.}
	\label{fig:run2_prev_results}
\end{figure}

\section{The CMS Run 2 Strategy}


Before discussing the full details of the analysis procedure and results, it is worth summarizing 
the broader strategy taken in our search for \hzg. As two prior CMS results have 
been published with Run 1 data and 2016 data, we will emphasize the ways in which our analysis
overlaps and differs from these previous approaches. Very broadly, the current search is 
similar to the previous analyses in trigger, object, and basic event selection. 
However, it is significantly more advanced in three body mass reconstruction, 
event categorization, and background modeling. We will show that the innovations in these 
areas have significantly improved the expected sensitivity and statistical robustness of the search 
with respect to the past CMS analyses.

Chapter 5 provides a detailed description of the data and Monte Carlo simulation 
used in our analysis. Standard dimuon and dielectron trigger streams are used for 2016, 2017, 
and 2018 LHC data. The full dataset corresponds to an integrated luminosity of 
137 $fb^{-1}$. In other words, we use the full CMS Run 2 dataset at 13 TeV center of 
mass energy. Simulated signal samples are used to determine the expected signal yields 
and three body mass shape. Simulated background samples are used for MVA training 
and category optimization. However, the background shape and normalization in the final result
is determined by fitting the data and does not rely on any simulation.

Chapter 6 describes the basic object and event selection used in the analysis, and Chapter 7 
describes further selection and categorization using MVAs. As mentioned, the basic 
selection requirements are fairly similar to previous CMS analyses. Muons are selected with 
a loose cut-based ID, while electrons and photons are selected with loose MVA IDs. Loose ID
requirements are chosen in order to maximize signal efficiency. Background is then suppressed 
through a combination of basic cuts and MVA methods. Basic kinematic cuts on isolation, mass, 
and photon energy variables are able to significantly reduce backgrounds from initial and final 
state radiation, while the MVA methods are able to strongly disciminate against backgrounds from 
jets misreconstructed as photons. Finally, a kinematic fit procedure in the dilepton mass 
is able to significantly improve the signal mass resolution. The kinematic fit is an 
innovation of the current analysis and contributes to its improved sensitivity.

Chapter 8 details the approach to signal and background modeling of the three body 
mass spectrum. As in previous CMS \hzg searches, the 
signal shape is determined via an analytic fit to simulation. A resonant background 
contribution from \hmumu is modeled similarly. The nonresonant 
background contribution is taken from a fit to data in the range of 105 to 170 GeV. This 
background model includes both the turn-on arising from the real Z boson peak as well as 
the falling spectrum at higher mass. We note that the turn-on was fit in the Run 1 analysis as 
well, but was dropped in the 2016 analysis. A more thorough discussion of the merits of fitting 
the turn-on will be described in chapter 8.

Chapter 9 describes the systematic uncertainties relevant for the analysis, and Chapter 10
gives a basic overview of the statistics used to arrive at the final results. It is 
worth noting that given the current integrated luminosity, the analysis is dominated by 
statistical uncertainty. Chapter 11 gives the full set of results, including best fit 
signal strength, limits, and comparisons with \hgg. 


